\section{Discussion}

Spring (April) and fall (November) each had 1.2 - 1.3 $\rm mg/m^3$, 0.9 - 1.0 $\rm mg/m^3$ of chlorophyll-a concentrations which was higher than the other seasons showing similarities with the result of Yamada, K., Ishizaka, J., Yoo, S., Kim, H. C. and Chiba, S. \cite{yamada2004seasonal}. Analysis of annual variability with the temporal resolution increased to 8-day-mean data showed winter (December, January, February) had higher concentration compared to summer (June, July, August), each with the values of 0.6 $\rm mg/m^3$ and $\rm mg/m^3$. It is assumed that the East Sea (Sea of Japan) shows clear seasonal differences.

In a perspective of spatial resolution, both the LAC data and the GAC data were suitable for research on the chlorophyll-a concentration variability in the East Sea (Sea of Japan) because both data showed similar tendency. However, for long-term research the GAC data is recommended since it has more consistent data. On the other hand, the LAC data with higher resolution is more likely to have an accurate data. 

Considering speckle errors should be also discussed. The LAC data could have more pixels over 10 $\rm mg/m^3$ compared to the GAC data because it has more speckles. LAC data has higher maximum value of chlorophyll-a concentration compared to the GAC data, and this could also be the effect of speckle errors. 

The limit of this research is that speckle errors were not corrected, which caused the concentration values to be higher than the in-situ data. Moreover, the in-situ data of the area of interest was not measured in this research, so it could not directly show whether the LAC data or the GAC data is more accurate. Further research will compare the processed data of the LAC and GAC with the in-situ data from previous researches.
