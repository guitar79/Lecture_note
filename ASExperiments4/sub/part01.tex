
%----------------------------------------------------------------------------------------
%	PART
%----------------------------------------------------------------------------------------

\part{대기과학}

%----------------------------------------------------------------------------------------
%	CHAPTER
%----------------------------------------------------------------------------------------

\chapterimage{chapter_head_2.pdf} % Chapter heading image

\chapter{대기과학 기초}


\section{과학 측정}

\subsection{목적}\index{목적}

과학자들이 과학기기들을 사용하는 이유, 차원과 측정 단위, 과학표기법, 백분율 오차에 대해 알아보고자 한다.

\subsection{관측과 측정}\index{관측과 측정}

과학자들의 임무 중의 많은 부분은 자연 세계의 관측(ovserfvation)\을 수행하는 것이 포함된다. 관측은 인간의 오감 중의 하나를 사용하여 간단히 이루어질 수 있다. 오감은 보고, 만지고, 냄새를 맡고, 맛을 보고, 듣는 것이다. 오늘날 과학자들은 과학기기(scientific instrument)들을 사용하여 그들 자신의 감각들을 확장한다. 측기들은 인간의 감각들을 확장하도록 고안되었고 자연 세계를 정확하게 측정하는 도구이다. 과학기기들에는 간단한 자(scale)에서부터 복잡한 위성 또는 전파 망원경까지 존재한다. 측정(measurement)은 과학의 중요한 부분이다. 왜냐하면, 측정은 관찰을 보다 정확하게 이루어지도록 하기 때문이다.
또한 정확한 측정은 지구의독측한 물리적인 또는 화학적인 성질들을 확인하는데 중요하다. 오늘날 과학자들은 자연 세계를 환찰하는 데 도움을 주는 4가지 기본적인 측정을 사용한다. 즉, 길이, 질량, 시간과 에너지이다. 길이(length)는 2개의 고정점들 사이의 거리이다. 


\section{대기과학에 사용되는 단위들}\index{Paragraphs of Text}



\part{Part Two}


%----------------------------------------------------------------------------------------
%	CHAPTER
%----------------------------------------------------------------------------------------

\chapter{실험}

\section{빗방울의 낙하 속도}\index{빗방울의 낙하 속도}

Rain is the liquid form of precipitation on Earth. It is part of the hydrologic cycle that begins when water evaporates and forms clouds in the atmosphere. The water that forms these clouds is frozen and vaporized. Once enough water has evaporated, it is then released in the form of droplets of rain back to the surface of the Earth.

The average size of a raindrop is 6 millimeters in diameter, about the size of a housefly. Of course all raindrops vary in size due to the strength of a specific rainstorm, but this is considered a reasonable value of a typical raindrop. When a raindrop falls to the surface of the Earth, it is acted on by two main forces, gravity and drag. A stationary raindrop initially experiences an acceleration due to gravity of 9.8 m/s2, as would any falling body. As gravity increases the speed of the raindrop in its descent, drag retards the downward acceleration of the raindrop. Usually, air resistance that comes in contact with the water molecules as they fall causes the drag. The combination of these two forces causes a raindrop to reach a terminal velocity when the drag force is approximately equal to the weight of the raindrop. At this point, a raindrop experiences no further acceleration and therefore falls at a constant velocity.

The magnitude of the terminal velocity of an object is also affected by its orientation. A common misconception is the shape of the raindrop. It is often depicted as pointy and lopsided. However, research has found the shape of a raindrop to be rather spherical or slightly flattened on the bottom by airflow like a hamburger bun.

The terminal velocity of a 6-millimeter raindrop was found to be approximately 10 m/s. This value has been found to vary between 9 m/s and 13 m/s when measurements were taken on different days. The variance has been contributed to different air temperatures and pressures. In comparison, a human being falling to the surface of the Earth experiences a drastically larger terminal velocity of approximately 56 m/s.


How Fast Is Falling Rain? \cite{evan2007}

Let the physics begin. You might think: hey, wont’ the speed depend on how high the water started? Well, it would if air resistance on the water drop were not important. However, I suspect that the rain will fall at terminal velocity. Terminal velocity is the case when the air resistance on the object is equal to the gravitational force on the object. When this happens, the net force is zero (the zero vector) and the object falls at a constant speed.

Here is a diagram of a water drop at terminal speed.

Untitled 1
Since the air resistance force depends on the speed of the object (but the gravitational force does not), there is one speed for which these two forces add up to the zero vector. Near the surface of the Earth, the magnitude of the gravitational force can be modeled as:

La te xi t 1 4

Where g is the local gravitational field (not the acceleration due to gravity – that is a non-good name for it). And what about the air resistance? It can probably be modeled as:

La te xi t 1 5

Where:

ρ is the density of air (about 1.2 kg/m3).
A is the cross-sectional area of the object. If the object was a sphere, this area would be the area of a circle.
C is the drag coefficient. This depends on the shape of the object. A cone and a flat circle will have the same A, but different drag coefficients.
v is the magnitude of the velocity of the object with respect to the air.
It won’t matter for this case too much, but the direction of the air resistance force is in the opposite direction to the velocity.
At terminal velocity, the magnitudes of these two forces will be equal. I can write that as:

La te xi t 1 6

Now, what about the mass (m)? Let me assume that it is made of water (like most rain) and is spherical (even though that is not likely – it would probably be “rain drop shaped”). If I call the density of water ρw and the radius of the drop r, then the mass would be:

La te xi t 1 7

Putting this into the “weight = air resistance” expression above as well as an expression for the cross-sectional area in terms of r, I get:

La te xi t 1 8

The cool thing here is that the terminal speed of the water drop depends on the size (radius). Larger drops will have a larger terminal velocity. So, could you just make a water melon sized water drop? No. Why not? Because at some point, the force from the air on the drop is going to break the water drop apart. The surface tension holding the drop together just won’t be enough to maintain its drop status.

Then how big can it get? I have no idea. Oh, and then there is the problem of real drop instead of spherical drops. Let me look at that first. Wikipedia lists the coefficient of drag for a smooth sphere as 0.1. A rain drop should be less than this – but how much less? Well, a rain drop would take some of the water to form some sort of tail. This would decrease the cross sectional area as well as decrease the drag coefficient. I am not sure how to calculate the volume of a non-spherical rain drop, so for now I will just use a spherical drop with a drag coefficient of 0.08. I know that is wrong, but it will give me an idea about the terminal speed.

Now, how big should it be? How about I don’t decide. Instead I will plot the terminal speed for a range of rain drop sizes. Let me look at drops from 0.5 mm to 5 mm. Here is that plot.

Raindrop.png

Well, the original question asked about the speeds in units of miles per hour. Here is the same plot but with different units.

Raindrop 2.png

Based on my estimations, 17 mph would be on the low end – but possible. It could be likely that I grossly overestimated the size of a raindrop.

Homework: Yes, there is homework. If the rain drop has a radius of 0.5 mm, from how high would it have to drop to get pretty close to the terminal velocity?


UPDATE


As usual, I rush into things without exploring things in more depth. My assumption of a raindrop shaped raindrop appears to be bogus. Who would have guessed that? Anyway, here are some very useful links from commenters (Jens and Charles) and a large thanks to them.

A German kid’s video showing the shape of a raindrop (I think).
A nice summary of findings for falling rain drops.
Terminal Velocity of Rain Drops Aloft – paper from the Journal of Applied Meteorology (pdf)
Here is another link from @swansontea: Bad Rain: Raindrops are not tear drop shaped.

%----------------------------------------------------------------------------------------
%	CHAPTER 
%----------------------------------------------------------------------------------------

\chapter{대기의 안정도와 단열선도}

\section{단열선도}
1930년대부터 고층기상관측이 실시되면서 대기를 입체적으로 파악하게 되었다. 고층기상관
측 자료를 신속하게 정리·분석하여 한 관측 지점의 연직방향 열역학적 특성을 파악하는데 단
열선도가 이용되고 있다. 단열선도는 복잡한 수식으로 계산해야 하는 기상 현상이나 기상요소
를 간단히 구할 수 있도록 고안된 도표이다. 단열선도에는 여러 종류가 있으나 어느 것이나 건
조 단열선, 습윤 단열선, 포화 혼합비선, 등온선, 등압선의 5선이 그려져 있다. 온도와 기압이
기울어져 Skew T-log P diagram라는 명명된 단열선도<그림 Ⅲ-18>이 가장 많이 쓰이고 있다.

\begin{figure}[h]
	\centering
	\includegraphics[width=0.8\linewidth]{skewT-logP-1}
	\caption{단열선도의 구성 선들}
	\label{fig:skewt-logp}
\end{figure}

단열선도에서 구할 수 있는 기상요소들은 다음과 같다.
① 혼합비 : 건조공기 1 kg 속에 혼합된 수증기의 질량(g)으로 Skew T-log P diagram에서
는 알고자 하는 고도의 이슬점 온도를 지나는 포화혼합비선의 값을 읽으면 된다.
② 상대습도 : R.H(%) = 실제 혼합비/ 포화 혼합비  ×100이므로 Skew T-log P diagram에서는
R.H(%)= 이슬점 온도에 해당하는 혼합비/ 기온에 해당하는 혼합비 
이 된다.
③ 상승응결고도(LCL, Lifting Condensation Level) : 온도점에서 건조 단열선을 따라 올라간
선과 이슬점에서 포화 혼합비선을 따라 올라간 선이 만나는 점의 고도이다.
④ 대류응결고도(CCL, Convection Condensation Level) : 상승응결고도에서 포화 혼합비선을
따라 올라간 선과 기온 곡선이 만나는 점의 고도이다


\section{사용 실례}\index{단열선도 사용 실례}

예를 들어 기압이 1000 hPa, 기온 3 ℃, 상대습도 70%인 공기덩어리(A점)가 위로 상승한다
고 가정하자. 기온 3 ℃에서의 포화혼합비를 그림에서 구하면 5 g이므로 이 공기의 혼합비는
그 70%인 3.5 g이다. 이 값은 공기 중의 수증기가 응결되어 줄어 들거나 또는 외부로부터 공
급되지 않는 한 변하지 않는다. 이 공기덩어리가 상층으로 올라가면 단열적으로 변화하므로
단열선도의 건조 단열선을 따라 기압이 내려간다. 그리고 925 hPa 부근에서 3.5 g의 혼합비선
과 만난다. 이것은 이 공기덩어리의 상대습도가 점차 증가하여 여기서 100% 즉, 포화에 도달
된다는 것을 의미한다. 따라서 그 이상의 고도로 올라가면 공기덩어리 중의 수증기가 응결하
면서 변화하게 되므로 기온이 습윤 단열선을 따라 변해간다. 수증기가 응결하기 시작하는 이
고도를 응결고도 또는 구름이 생성되는 높이이므로 운저고도(雲底高度), 운고라 한다. 이 예에
서 응결고도는 925 hPa 고도이며 대략 지상 750 m 정도가 된다.

\begin{figure}[h]
	\centering
	\includegraphics[width=0.8\linewidth]{skewT-logP-2}
	\caption{단열선도 이용 실례}
	\label{fig:skewt-logp-1}
\end{figure}


\section{대기 안정도 분석}\index{단열선도를 이용한 대기 안정도 분석}

단열선도를 이용하면 대기층의 안정 상태를 쉽게 파악할 수 있다. 고층관측 결과를 단열선
도에 기입했을 때 대기의 온도가 <그림 Ⅲ-19>의 굵은 선 B와 같이 변하였다고 하자. 지상의
공기덩어리를 상층으로 가지고 올라갔을 때의 상황을 조사해 보면, 그림에서 파선과 같이 되
므로 이 공기덩어리의 기온은 주위 공기의 기온보다 높아져 밀도가 작아지므로 부력을 받아
계속 상승하려 한다. 그래서 이런 상태의 대기성층을 불안정이라 한다. 이런 때는 대류가 일
어난다. 여기서 알 수 있는 것처럼 대기의 실제 기온감률이 100 m당 1 ℃ 이상이면 그 기층은
불안정하다는 것을 알 수 있으며, 특히 이 경우를 절대 불안정이라 한다.
대기의 온도 변화가 <그림 Ⅲ-19>에서 굵은 선 C와 같았다고 할 경우에는, 그 기층의 공기
가 포화되지 않고 건조하다면 건조 단열선을 따라 변하므로 상공으로 올라가면서 밀도가 본래
의 대기 밀도보다 커지므로 안정하게 된다. 그런데 공기가 포화되어 있으면 습윤 단열선을 따
라 변하므로 상공에 올라가 주위 공기보다 기온이 높아져 불안정해진다. 이상으로 보면 수증
기 함유량에 따라 안정하기도 하고 불안정하기도 한다. 이것을 조건부 불안정이라 한다.
<그림 Ⅲ-19>에서 D와 같은 경우에는 공기가 건조한 경우나 습윤한 경우에 관계 없이 위로
올라간 공기덩어리는 주위 공기보다 기온이 낮아 안정적이다. 이것을 기온감률로 표현하면 실
제 기온감률이 습윤단열변화 할 때보다 작은 경우로서 대기의 기온감률이 100 m당 0.5 ℃ 미
만인 경우이다. 이를 절대 안정이라 한다.
만일 실제 기온의 고도변화가 100 m 당 1 ℃라면 공기덩어리가 건조단열 변화하면서 상승
해도 주위와의 기온차가 생기지 않는다. 또 기온의 고도변화가 습윤 단열선을 따르는 상태에
서는 포화된 공기덩어리는 상승해도 주위와의 기온차가 없다. 즉, 안정도 불안정도 아닌 중립
상태가 된다.
<그림 Ⅲ-19>의 E의 경우는 파선을 따라 변화한다. 따라서 이 경우 885hPa이하에서는 안정
이나 그 이상 끌어올리면 주위보다 기온이 높아져 불안정이 된다. 즉 요란이 작으면 안정 상태
를 유지하지만, 요란이 크면 불안정하게 된다. 이러한 상태를 잠재 불안정이라 한다.


%------------------------------------------------



\begin{definition}[단열선도를 이용한 대기 분석]


열선도 중에서 가장 널리 이용되고 있는 Skew T-log P diagram 상에 실제 관측 자료를 표시하여 여러 가지 기상현상을 구해보고, 이를 통해 대기를 입체적으로 분석해 보자.

1. 다음 표는 고층 기상 관측 자료이다.


\begin{tabular}{c|c|c|c|c|c}
	\hline 
기압(hPa) & 기온(℃) & 이슬점  온도 & 혼합비(g/kg) & 포화혼합비 & 상대습도(\%) \\ \hline 
1000 & 25 & 14 &	 &  &  \\ 
	\hline 
900 & 18 & 6 &  &  &  \\ 
	\hline 
800 & 8 & -1 &  &  &  \\ 
	\hline 
700 & 3 & -12  &  &  &  \\ 
	\hline 
600 & 0 & -10 &  &  &  \\ 
	\hline 
500 & -13 & -151 &  &  &  \\ 
	\hline 
400 & -16 &-40	&  &  &  \\ 
	\hline 
\end{tabular} 


	\end{align}
\end{definition}

%------------------------------------------------

\section{일기도 그리기}\index{일기도 그리기}

\subsection{일기도}\index{일기도}
일기도(지상 일기도)는 어느 지역 내의 일기 개황을 한 눈에 보아서 알 수 있도록
각종 기상요소(기압, 습도, 기온, 이슬점, 운량, 풍향, 풍속 등)를 나타낸 것으로 이것을
이용하여 각 지역의 일기를 알 수 있으며, 연속된 일기도를 통해 앞으로의 일기를 예상할 수 있다.

\subsection{관측자료 기입}\index{관측자료 기입}
일기도에 관측된 자료를 나타낼 때에는 <그림 Ⅲ-20>과 같이 정해진 일정한 형식으로 기입해야 한다. 기입이 끝나면 일기도 상에서 기압, 기온 또는 필요한 기상요소들에 대하여 값이 같은 점을 연결하여 선을 긋는다. 일기도에서 가장 중요하게 다루는 선이 등압선이다. 등압선은 기압이 같은 지점을 연결한 선이다. 등압선을 그려보면 고기압과 저기압의 위치, 전선 등을 가려 낼 수 있다. 이것은 마치 지도에 그려진 등고선의 분포모양을 보고 산이나 골짜기를 판별하는 것과 유사하다. 등압선을 그리는 방법은 다음과 같다.
① 주어진 일기도에 기입된 각종 자료 중 구름의 양, 풍향, 풍속, 기압 등의 의미를 파악한다.
기압은 자연 상태에서 보통 1060 hPa을 넘지 않으므로, 첫자리 수가 0에서 5사이이면 10
을, 6에서 9사이이면 9를 각각 앞에 붙여 계산한다.
② 일반적으로 등압선은 1000 hPa을 기준으로 ......, 992, 996, 1000, 1004, 1008, ...... 등 4 hPa 간격으로 그린다. 그러나 등압선 간의 폭이 너무 넓어 기압배치를 파악하기 어려울
때는 2 hPa 간격으로 파선을 그린다.
③ 그리기 쉬운 곳(등압선이 조밀하지 않은 곳)부터 그려 나간다.
④ 등압선은 중간에서 끊어지거나 없어지지 않는다.
⑤ 관측값이 없는 경우는 내삽, 또는 외삽법의 원리로 <그림 Ⅲ-21>과 같이 이웃하는 두 지
점의 간격을 비례로 나누어서 부드럽고 매끈한 곡선으로 그린다.
⑥ 한 선으로 연결되는 등압선은 양쪽 끝에 기압의 값을 기입하고, 폐곡선의 경우는 위쪽(북
쪽) 중앙에 등압선을 끊고 값을 기입한다. 저기압의 중심은 적색으로 L(low pressure), 고
기압의 중심은 청색으로 H(high pressure)라고 표시한다.
일기도가 그려지면 고기압과 저기압의 위치 및 이동경로, 기압과 고도의 변화경향, 전선의
발생 및 소멸과 이동, 날씨변화, 대기의 수직구조 등을 세밀히 분석하게 된다. 전선의 위치를
찾는 방법은 다음과 같다.
\begin{figure}
	\centering
	\includegraphics[width=0.8\linewidth]{Pictures/draw-weathermap01}
	\caption{등압선의 작도 방법}
	\label{fig:draw-weathermap01}
\end{figure}법

① 전선은 기온, 이슬점, 풍향이 불연속을 이루므로 이들 값이 급변하는 지역을 찾는다.
② 전선 상에서는 일반적으로 일기가 악화되므로 강수 등 나쁜 일기가 선상으로 나타날 경
우 전선이 존재할 가능성이 높다. 온난전선은 전면의 넓은 지역에서 강수 현상이 나타나
고 후면에는 비교적 맑은 날씨를 이룬다. 한랭전선은 전면과 후면의 구별 없이 전선 상에
서 비교적 좁은 지역에 강수 현상이 나타난다.
③ 전선이 확인되면 <그림 Ⅲ-22>와 같이 등압선이 휘도록 연결한다.

\begin{figure}
	\centering
	\includegraphics[width=0.8\linewidth]{Pictures/draw-weathermap02}
	\caption{전선에서의 등압선}
	\label{fig:draw-weathermap02}
\end{figure}



일기도 그리기
여러 가지 기상요소 중 간단한 등압선을 그리고 고·저기압의 위치 및 전선을 찾아 일기도를 완성해 보자.




\section{Corollaries}\index{Corollaries}

This is an example of a corollary.

\begin{corollary}[Corollary name]
	The concepts presented here are now in conventional employment in mathematics. Vector spaces are taken over the field $\mathbb{K}=\mathbb{R}$, however, established properties are easily extended to $\mathbb{K}=\mathbb{C}$.
end{corollary}

%------------------------------------------------

\section{일기도 분석}\index{일기도 분석}

기온, 강수 유무 등의 매일의 일기는 산업과 일상생활에 많은 영향을 미친다. 따라서 오랜
세월 동안 일기를 정확히 예측하고자 많은 노력을 기울여 왔다. 아직도 많은 한계를 안고 있으
나 날씨는 갑자기 변하는 것이 아니고 과거로부터 연속성을 가지고 변화하므로 과거와 현재의
일기상태를 분석함으로써 미래의 일기를 어느 정도 예측할 수 있으며 보다 정확한 예측을 위
하여 노력을 계속하고 있는 실정이다.
일기의 분포는 일반적으로 기압배치에 대응되므로 기압배치를 예상할 수 있으면 일기예보
가 어느 정도 가능하다. 고기압, 저기압의 기압계는 지속성을 가지고 있고, 우리나라 주변에서
는 서에서 동으로 이동하므로 이를 이용하여 앞으로의 기압배치를 예측하고 날씨를 예상할 수
있다. 고기압이 있는 지역의 지상일기는 하강 기류로 인해 구름이 소멸되어 맑은 날씨를 나타
낸다. 반면 저기압지역에서는 내부의 상승기류로 인해 구름이 만들어지고 비가 내리게 되어
궂은 날씨가 된다. 저기압은 전선을 동반하는 경우도 있는데 전선은 성질이 다른 두 기단의 경
계를 이루는 상대적으로 좁은 영역을 이루므로 이 영역을 기준으로 온도나 바람 등이 급변할
뿐 아니라 일반적으로 일기가 악화되어 강수 현상 등이 나타난다.

\begin{figure}
	\centering
	\includegraphics[width=0.8\linewidth]{Pictures/weathermap01}
	\caption{태풍이 이동하는 일기도}
	\label{fig:weathermap01}
\end{figure}


실제로 일기도가 완성되면 예보관은 이를 다각적으로 분석한다. 일기도의 분석은 고기압과
저기압의 위치 및 이동경로, 기압과 고도의 변화경향, 전선의 발생 및 소멸과 이동 추적, 날씨
변화, 대기의 연직구조 등을 세밀히 분석한다. 또한 특수기상 관측 자료인 기상레이더에 의한
강수구역 추적, 기상위성에 의한 구름사진 분석, 자동기상관측자료(AWS) 등을 분석하여 앞으
로의 날씨변화를 예상하게 된다. 또한 최근에는 컴퓨터를 활용하여 짧은 시간에 많은 기상자
료를 처리할 수 있게 됨으로써 수치분석자료와 각종 보조일기도들을 이용할 수 있게 되어 정
확한 날씨를 판단하는데 많은 도움을 주고 있다.

\subsection{예상일기도 그리기}\index{Propositions!Several Equations}

연속적인 몇 장의 일기도를 분석하여 앞으로 전개될 대기 상태를 예측하여 예상 일기도를 그려 보자.

1. <그림 1>~<그림 4>는 2007년 7월 1일 6시부터 2일 18시까지 12시간 간격으로 연속 4회 동안 관
측하여 작성한 지상일기도이다.
⑴ 일기도 상에 나타난 전선을 다음 그림에 그려 넣고, 전선이 어느 방향으로 얼마나 이동했으며 평
균 이동 속도는 얼마인지 다음 표를 작성하라. (단, 경도 1°의 거리는 위도 30°에서 96.5 km, 위도
40°에서 85.4 km, 위도 50°에서 71.7 km이다. 위도 1°사이의 거리는 약 110 km이다.)


⑵ 7월 3일 6시에는 기압계가 어떻게 달라졌을지 예상하여 다음 그림에 예상 일기도를 그려 보자.

⑶ 다음 표는  기상청에서 발표한 7월 1일과 2일의 예보 통보문이다. 그린 예상 일기도를 바탕으로 예보 통보문을 작성해 보자.

\begin{tabular}{|c|c|}
	\hline 
일시	& 예보 통보문 \\ 
	\hline 
	7월 1일	& 동해상에 위치한 고기압 가장자리에 들겠습니다.
	전국이 대체로 구름 많고, 경상북도 지방은 새벽까지, 전라남북도 지방에서 아침까지 소
	나기(강수확률 60{\%})가 오는 곳이 있겠고, 오후에는 대기불안정으로 남부 내륙지방을 중
	심으로 산발적으로 소나기(강수확률 60{\%}가 오는 곳이 있겠습니다. \\ 
	\hline 
	7월 2일	& 서쪽에서 접근하는 장마전선의 영향을 점차 받겠습니다.
	전국이 대체로 흐리고 새벽에 중부서해안지방부터 비(강수확률 60~ 90{\%}가 시작되어 밤
	에는 전국으로 확대되겠습니다. \\ 
	\hline 
	7월 3일	&  \\ 
	\hline 
\end{tabular} 


\chapterimage{chapter_head_1.pdf} % Chapter heading image

\chapter{일기예보 과정}

\section{예보가 나오는 과정}

예보가 나오기까지는 기상실황파악 → 자료수집 → 분석 → 예보작성 → 통보 과정을 거친다. 

%------------------------------------------------

\section{기상실황 파악}

\subsection{지상기상 관측}
전국 76개소의 기상관서에서 하늘상태, 시정 등의 목측 (目測 ) 요소를 관측하고 있으며, 기온, 습도, 강수량, 바람, 기압 등은 자동기상관측장비를 이용하여 1분 간격으로 관측되고 있다. 또한, 기상관서가 없는 500여 소에서는 방재용 자동기상관측장비를 이용하여 기온, 풍향, 풍속, 강수량, 강수 유무를 1분 간격으로 관측하여 기상실황을 감시하고 있다.

\subsection{항공기상관측}
전국 공항기상관서에서는 바람, 시정, 운고, 기온, 기압 등의 항공기상관측요소를 매30분 또는 1시간 간격으로 관측하며, 활주로에 설치된 공항기상관측장비에 의해 기상요소들이 매분 자동 관측된다. 특히 인천, 제주, 양양, 울산 등의 공항에서는 이·착륙 항공기에 영향을 미치는 저층난류를 탐측하기 위하여 저층난류경보장치를 운영하고 있다.

\subsection{고층기상관측}
고층기상관측은 지상보다 높은 상층 대기의 상태를 관측하는 것으로, 기상청은 레윈존데 관측, 수직측풍장비 관측을 수행한다. 레윈존데 관측은 기구에 라디오존데를 매달아 지상으로부터 약 35 km(5 hPa)까지의 고도별 기압, 기온, 습도, 풍향, 풍속을 00UTC와 12UTC에 관측하며, 수직측풍장비 관측은 UHF나 VHF 파장의 전파를 상층대기로 방사하고 바람과 함께 이동하는 난류에 산란되어 다시 수신되는 전파신호로 바람을 10분 간격으로 관측한다.

\subsection{해양기상관측}
해양기상관서, 해양기상관측부이, 해양기상영상감시시스템을 통하여 풍향·풍속, 기온, 수온, 기압, 파고 등을 관측하며, 먼바다의 기상현상 관측 및 부이 관리를 위하여 기상관측선을 운영하고 있다.

\subsection{기상위성관측}
기상위성은 우주공간에서 지구의 기상변화를 관측한다. 기상청은 정지궤도기상위성과 극궤도기상위성의 자료를 직접 수신하여 처리하여 예보를 위해 사용하고, 국민에게도 공개하고 있다. 이를 위해 기상위성 수신처리분석시스템을 서울, 문산, 서산에 설치하여 운영중이다.

\subsection{기상레이더 관측}
도플러 기상레이더를 설치하여 한반도에서 발생하는 악기상을 관측하여 예보에 활용하고 있다. 또한 일본 기상청과 공군의 레이더 자료도 수신하여 기존영상과 합성하여 종합적으로 활용하고 있다.


\section{자료수집}
통신용컴퓨터를 이용하여 국내기상자료와 외국에서 송신되는 각종기상자료를 수집, 편집, 가공하여 분석용 컴퓨터로 보낸다. 국내·외에서 수집된 관측자료로부터 수치예보모델을 이용하여 예상일기도를 생산한다. 이러한 수치예보모델의 운용을 위해 슈퍼컴퓨터가 사용된다.


%----------------------------------------------------------------------------------------
%	CHAPTER 
%----------------------------------------------------------------------------------------

\chapterimage{chapter_head_1.pdf} % Chapter heading image

\chapter{기상청(www.kma.go.kr) 일기 예보}

\section{기상청 일기 예보의 종류}


\subsection{초단기 동네 예보}
초단기예보는 현재부터 앞으로 3시간까지, 실황(날씨, 기온, 습도 등 7개 요소)과 예보(강수형태, 하늘상태, 강수량 등 3개 요소)를 1시간 간격으로 동네예보를 기반으로 매 시각 30분에 발표한다. 초단기예보는 짧은 시간에 발생·소멸하는 기상현상에 대해 신속하게 대응하여 재해예방에 최선을 다하고자 2010년 6월 15일부터 홈페이지를 통해 제공하고 있다.

\subsection{동네 예보}
대상기간과 구역을 시ㆍ공간적으로 세분화하여 발표하는 예보로 기온, 최고기온, 최저기온, 강수형태, 강수확률, 12시간강수량, 12시간적설, 하늘상태, 습도, 풍향, 풍속, 파고 등을 예보한다. 동네예보는 3시간 간격으로 1일에 8회 예보하며 예보구간도 역시 3시간 단위로 48시간까지 예보한다. 

\subsection{주간 예보}
기상전망, 예보구역별 육상 및 해상 날씨, 지점별 기온, 파고에 대한 48시간 이후부터  6일간의 예보로 일 2회 발표(06시, 18시)하는 주간예보(모레부터 6일간)가 계속 유지될 가능성에 대한 신뢰도 정보를 3단계로 구분하여 제공(육상)한다.

\begin{table}[h]
\centering
	\caption{신뢰도와 의미}
\begin{tabular*}{.8\linewidth}{c|c}
	\hline 
신뢰도  &	내용	  \\ 	\hline 
높음  & 다음날 발표 주간예보가 계속 유지될 가능성이 높음  \\  \hline 
보통 & 다음날 발표 주간예보가 계속 유지될 가능성이 있음  \\ 	\hline 
낮음 & 다음날 발표 주간예보가 계속 유지될 가능성이 낮음   \\ 	\hline 
\end{tabular*} 
\end{table}

\subsection{주말 예보}
토요일과 일요일의 기상 개황, 일별 날씨, 야외활동 지수 등의 정보를 제공한다. 화요일 19시부터 금요일 24시까지 제공되며, 매일 19시에 발표한다.

\subsection{장기 예보}
장기예보는 11일 이상에 대한 예보를 일컬으며 순별·월별 기압계 동향 및 전망, 기온·강수량 예보 등을 발표한다. 예보구역은 한반도 12개 권역(서울·인천·경기도, 강원도 영서, 강원도 영동, 대전·충청남도, 충청북도, 광주·전라남도, 전라북도, 부산·울산·경상남도, 대구·경상북도, 제주도, 평안남북도·황해도, 함경남북도)이며, 월 3회 발표되는 1월 전망과 월 1회 발표되는 3개월 전망이 있다. 그 외 연 4회 발표되는 기후전망은 다음다음 계절에 대한 전망으로 봄철 기후전망은 11월 23일 경에, 여름철 기후전망은 2월 23일 경에, 가을철 기후전망은 5월 23일 경에 겨울철 기후전망은 8월 23일 경에 발표한다.

