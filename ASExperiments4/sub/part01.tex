
%----------------------------------------------------------------------------------------
%	PART
%----------------------------------------------------------------------------------------

\part{대기과학 실험을 위한 준비}

%----------------------------------------------------------------------------------------
%	CHAPTER
%----------------------------------------------------------------------------------------

\chapterimage{chapter_head_2.pdf} % Chapter heading image

\chapter{대기과학 기초}\index{대기과학 기초}


\section{과학 측정}\index{과학 측정}

\subsection{목적}\index{목적}

과학자들이 과학기기들을 사용하는 이유, 차원과 측정 단위, 과학표기법, 백분율 오차에 대해 알아보고자 한다.

\subsection{관측과 측정}\index{관측과 측정}

과학자들의 임무 중의 많은 부분은 자연 세계의 관측(ovserfvation)\을 수행하는 것이 포함된다. 관측은 인간의 오감 중의 하나를 사용하여 간단히 이루어질 수 있다. 오감은 보고, 만지고, 냄새를 맡고, 맛을 보고, 듣는 것이다. 오늘날 과학자들은 과학기기(scientific instrument)들을 사용하여 그들 자신의 감각들을 확장한다. 측기들은 인간의 감각들을 확장하도록 고안되었고 자연 세계를 정확하게 측정하는 도구이다. 과학기기들에는 간단한 자(scale)에서부터 복잡한 위성 또는 전파 망원경까지 존재한다. 측정(measurement)은 과학의 중요한 부분이다. 왜냐하면, 측정은 관찰을 보다 정확하게 이루어지도록 하기 때문이다.
또한 정확한 측정은 지구의독측한 물리적인 또는 화학적인 성질들을 확인하는데 중요하다. 오늘날 과학자들은 자연 세계를 환찰하는 데 도움을 주는 4가지 기본적인 측정을 사용한다. 즉, 길이, 질량, 시간과 에너지이다. 길이(length)는 2개의 고정점들 사이의 거리이다. 


\section{대기과학에 사용되는 단위들}\index{Paragraphs of Text}

