
%----------------------------------------------------------------------------------------
%	PART
%----------------------------------------------------------------------------------------

\part{대기과학}

%----------------------------------------------------------------------------------------
%	CHAPTER
%----------------------------------------------------------------------------------------

\chapterimage{chapter_head_2.pdf} % Chapter heading image

\chapter{대기과학 기초}


\section{과학 측정}

\subsection{목적}\index{목적}

과학자들이 과학기기들을 사용하는 이유, 차원과 측정 단위, 과학표기법, 백분율 오차에 대해 알아보고자 한다.

\subsection{관측과 측정}\index{관측과 측정}

과학자들의 임무 중의 많은 부분은 자연 세계의 관측(ovserfvation)\을 수행하는 것이 포함된다. 관측은 인간의 오감 중의 하나를 사용하여 간단히 이루어질 수 있다. 오감은 보고, 만지고, 냄새를 맡고, 맛을 보고, 듣는 것이다. 오늘날 과학자들은 과학기기(scientific instrument)들을 사용하여 그들 자신의 감각들을 확장한다. 측기들은 인간의 감각들을 확장하도록 고안되었고 자연 세계를 정확하게 측정하는 도구이다. 과학기기들에는 간단한 자(scale)에서부터 복잡한 위성 또는 전파 망원경까지 존재한다. 측정(measurement)은 과학의 중요한 부분이다. 왜냐하면, 측정은 관찰을 보다 정확하게 이루어지도록 하기 때문이다.
또한 정확한 측정은 지구의독측한 물리적인 또는 화학적인 성질들을 확인하는데 중요하다. 오늘날 과학자들은 자연 세계를 환찰하는 데 도움을 주는 4가지 기본적인 측정을 사용한다. 즉, 길이, 질량, 시간과 에너지이다. 길이(length)는 2개의 고정점들 사이의 거리이다. 


\section{대기과학에 사용되는 단위들}\index{Paragraphs of Text}


%----------------------------------------------------------------------------------------
%	CHAPTER
%----------------------------------------------------------------------------------------

\chapter{실험}

\section{빗방울의 낙하 속도}\index{빗방울의 낙하 속도}

Rain is the liquid form of precipitation on Earth. It is part of the hydrologic cycle that begins when water evaporates and forms clouds in the atmosphere. The water that forms these clouds is frozen and vaporized. Once enough water has evaporated, it is then released in the form of droplets of rain back to the surface of the Earth.

The average size of a raindrop is 6 millimeters in diameter, about the size of a housefly. Of course all raindrops vary in size due to the strength of a specific rainstorm, but this is considered a reasonable value of a typical raindrop. When a raindrop falls to the surface of the Earth, it is acted on by two main forces, gravity and drag. A stationary raindrop initially experiences an acceleration due to gravity of 9.8 m/s2, as would any falling body. As gravity increases the speed of the raindrop in its descent, drag retards the downward acceleration of the raindrop. Usually, air resistance that comes in contact with the water molecules as they fall causes the drag. The combination of these two forces causes a raindrop to reach a terminal velocity when the drag force is approximately equal to the weight of the raindrop. At this point, a raindrop experiences no further acceleration and therefore falls at a constant velocity.

The magnitude of the terminal velocity of an object is also affected by its orientation. A common misconception is the shape of the raindrop. It is often depicted as pointy and lopsided. However, research has found the shape of a raindrop to be rather spherical or slightly flattened on the bottom by airflow like a hamburger bun.

The terminal velocity of a 6-millimeter raindrop was found to be approximately 10 m/s. This value has been found to vary between 9 m/s and 13 m/s when measurements were taken on different days. The variance has been contributed to different air temperatures and pressures. In comparison, a human being falling to the surface of the Earth experiences a drastically larger terminal velocity of approximately 56 m/s.


How Fast Is Falling Rain? \cite{evan2007}

Let the physics begin. You might think: hey, wont’ the speed depend on how high the water started? Well, it would if air resistance on the water drop were not important. However, I suspect that the rain will fall at terminal velocity. Terminal velocity is the case when the air resistance on the object is equal to the gravitational force on the object. When this happens, the net force is zero (the zero vector) and the object falls at a constant speed.

Here is a diagram of a water drop at terminal speed.

Untitled 1
Since the air resistance force depends on the speed of the object (but the gravitational force does not), there is one speed for which these two forces add up to the zero vector. Near the surface of the Earth, the magnitude of the gravitational force can be modeled as:

La te xi t 1 4

Where g is the local gravitational field (not the acceleration due to gravity – that is a non-good name for it). And what about the air resistance? It can probably be modeled as:

La te xi t 1 5

Where:

ρ is the density of air (about 1.2 kg/m3).
A is the cross-sectional area of the object. If the object was a sphere, this area would be the area of a circle.
C is the drag coefficient. This depends on the shape of the object. A cone and a flat circle will have the same A, but different drag coefficients.
v is the magnitude of the velocity of the object with respect to the air.
It won’t matter for this case too much, but the direction of the air resistance force is in the opposite direction to the velocity.
At terminal velocity, the magnitudes of these two forces will be equal. I can write that as:

La te xi t 1 6

Now, what about the mass (m)? Let me assume that it is made of water (like most rain) and is spherical (even though that is not likely – it would probably be “rain drop shaped”). If I call the density of water ρw and the radius of the drop r, then the mass would be:

La te xi t 1 7

Putting this into the “weight = air resistance” expression above as well as an expression for the cross-sectional area in terms of r, I get:

La te xi t 1 8

The cool thing here is that the terminal speed of the water drop depends on the size (radius). Larger drops will have a larger terminal velocity. So, could you just make a water melon sized water drop? No. Why not? Because at some point, the force from the air on the drop is going to break the water drop apart. The surface tension holding the drop together just won’t be enough to maintain its drop status.

Then how big can it get? I have no idea. Oh, and then there is the problem of real drop instead of spherical drops. Let me look at that first. Wikipedia lists the coefficient of drag for a smooth sphere as 0.1. A rain drop should be less than this – but how much less? Well, a rain drop would take some of the water to form some sort of tail. This would decrease the cross sectional area as well as decrease the drag coefficient. I am not sure how to calculate the volume of a non-spherical rain drop, so for now I will just use a spherical drop with a drag coefficient of 0.08. I know that is wrong, but it will give me an idea about the terminal speed.

Now, how big should it be? How about I don’t decide. Instead I will plot the terminal speed for a range of rain drop sizes. Let me look at drops from 0.5 mm to 5 mm. Here is that plot.

Raindrop.png

Well, the original question asked about the speeds in units of miles per hour. Here is the same plot but with different units.

Raindrop 2.png

Based on my estimations, 17 mph would be on the low end – but possible. It could be likely that I grossly overestimated the size of a raindrop.

Homework: Yes, there is homework. If the rain drop has a radius of 0.5 mm, from how high would it have to drop to get pretty close to the terminal velocity?


UPDATE


As usual, I rush into things without exploring things in more depth. My assumption of a raindrop shaped raindrop appears to be bogus. Who would have guessed that? Anyway, here are some very useful links from commenters (Jens and Charles) and a large thanks to them.

A German kid’s video showing the shape of a raindrop (I think).
A nice summary of findings for falling rain drops.
Terminal Velocity of Rain Drops Aloft – paper from the Journal of Applied Meteorology (pdf)
Here is another link from @swansontea: Bad Rain: Raindrops are not tear drop shaped.

%----------------------------------------------------------------------------------------
%	CHAPTER 
%----------------------------------------------------------------------------------------

\chapter{In-text Elements}

\section{Theorems}\index{Theorems}

This is an example of theorems.

\subsection{Several equations}\index{Theorems!Several Equations}
This is a theorem consisting of several equations.

\begin{theorem}[Name of the theorem]
	In $E=\mathbb{R}^n$ all norms are equivalent. It has the properties:
	\begin{align}
		& \big| ||\mathbf{x}|| - ||\mathbf{y}|| \big|\leq || \mathbf{x}- \mathbf{y}||\\
		&  ||\sum_{i=1}^n\mathbf{x}_i||\leq \sum_{i=1}^n||\mathbf{x}_i||\quad\text{where $n$ is a finite integer}
	\end{align}
\end{theorem}

\subsection{Single Line}\index{Theorems!Single Line}
This is a theorem consisting of just one line.

\begin{theorem}
	A set $\mathcal{D}(G)$ in dense in $L^2(G)$, $|\cdot|_0$. 
\end{theorem}

%------------------------------------------------

\section{Definitions}\index{Definitions}

This is an example of a definition. A definition could be mathematical or it could define a concept.

\begin{definition}[Definition name]
	Given a vector space $E$, a norm on $E$ is an application, denoted $||\cdot||$, $E$ in $\mathbb{R}^+=[0,+\infty[$ such that:
	\begin{align}
		& ||\mathbf{x}||=0\ \Rightarrow\ \mathbf{x}=\mathbf{0}\\
		& ||\lambda \mathbf{x}||=|\lambda|\cdot ||\mathbf{x}||\\
		& ||\mathbf{x}+\mathbf{y}||\leq ||\mathbf{x}||+||\mathbf{y}||
	\end{align}
\end{definition}

%------------------------------------------------

\section{Notations}\index{Notations}

\begin{notation}
	Given an open subset $G$ of $\mathbb{R}^n$, the set of functions $\varphi$ are:
	\begin{enumerate}
		\item Bounded support $G$;
		\item Infinitely differentiable;
	\end{enumerate}
	a vector space is denoted by $\mathcal{D}(G)$. 
\end{notation}

%------------------------------------------------

\section{Remarks}\index{Remarks}

This is an example of a remark.

\begin{remark}
	The concepts presented here are now in conventional employment in mathematics. Vector spaces are taken over the field $\mathbb{K}=\mathbb{R}$, however, established properties are easily extended to $\mathbb{K}=\mathbb{C}$.
\end{remark}

%------------------------------------------------

\section{Corollaries}\index{Corollaries}

This is an example of a corollary.

\begin{corollary}[Corollary name]
	The concepts presented here are now in conventional employment in mathematics. Vector spaces are taken over the field $\mathbb{K}=\mathbb{R}$, however, established properties are easily extended to $\mathbb{K}=\mathbb{C}$.
\end{corollary}

%------------------------------------------------

\section{Propositions}\index{Propositions}

This is an example of propositions.

\subsection{Several equations}\index{Propositions!Several Equations}

\begin{proposition}[Proposition name]
	It has the properties:
	\begin{align}
		& \big| ||\mathbf{x}|| - ||\mathbf{y}|| \big|\leq || \mathbf{x}- \mathbf{y}||\\
		&  ||\sum_{i=1}^n\mathbf{x}_i||\leq \sum_{i=1}^n||\mathbf{x}_i||\quad\text{where $n$ is a finite integer}
	\end{align}
\end{proposition}

\subsection{Single Line}\index{Propositions!Single Line}

\begin{proposition} 
	Let $f,g\in L^2(G)$; if $\forall \varphi\in\mathcal{D}(G)$, $(f,\varphi)_0=(g,\varphi)_0$ then $f = g$. 
\end{proposition}

%------------------------------------------------

\section{Examples}\index{Examples}

This is an example of examples.

\subsection{Equation and Text}\index{Examples!Equation and Text}

\begin{example}
	Let $G=\{x\in\mathbb{R}^2:|x|<3\}$ and denoted by: $x^0=(1,1)$; consider the function:
	\begin{equation}
		f(x)=\left\{\begin{aligned} & \mathrm{e}^{|x|} & & \text{si $|x-x^0|\leq 1/2$}\\
			& 0 & & \text{si $|x-x^0|> 1/2$}\end{aligned}\right.
	\end{equation}
	The function $f$ has bounded support, we can take $A=\{x\in\mathbb{R}^2:|x-x^0|\leq 1/2+\epsilon\}$ for all $\epsilon\in\intoo{0}{5/2-\sqrt{2}}$.
\end{example}

\subsection{Paragraph of Text}\index{Examples!Paragraph of Text}

\begin{example}[Example name]
	\lipsum[2]
\end{example}

%------------------------------------------------

\section{Exercises}\index{Exercises}

This is an example of an exercise.

\begin{exercise}
	This is a good place to ask a question to test learning progress or further cement ideas into students' minds.
\end{exercise}

%------------------------------------------------

\section{Problems}\index{Problems}

\begin{problem}
	What is the average airspeed velocity of an unladen swallow?
\end{problem}

%------------------------------------------------

\section{Vocabulary}\index{Vocabulary}

Define a word to improve a students' vocabulary.

\begin{vocabulary}[Word]
	Definition of word.
\end{vocabulary}

%----------------------------------------------------------------------------------------
%	PART
%----------------------------------------------------------------------------------------

\part{Part Two}

%----------------------------------------------------------------------------------------
%	CHAPTER 
%----------------------------------------------------------------------------------------

\chapterimage{chapter_head_1.pdf} % Chapter heading image

\chapter{일기예보 과정}

\section{예보가 나오는 과정}

예보가 나오기까지는 기상실황파악 → 자료수집 → 분석 → 예보작성 → 통보 과정을 거친다. 

%------------------------------------------------

\section{기상실황 파악}

\subsection{지상기상 관측}
전국 76개소의 기상관서에서 하늘상태, 시정 등의 목측 (目測 ) 요소를 관측하고 있으며, 기온, 습도, 강수량, 바람, 기압 등은 자동기상관측장비를 이용하여 1분 간격으로 관측되고 있다. 또한, 기상관서가 없는 500여 소에서는 방재용 자동기상관측장비를 이용하여 기온, 풍향, 풍속, 강수량, 강수 유무를 1분 간격으로 관측하여 기상실황을 감시하고 있다.

\subsection{항공기상관측}
전국 공항기상관서에서는 바람, 시정, 운고, 기온, 기압 등의 항공기상관측요소를 매30분 또는 1시간 간격으로 관측하며, 활주로에 설치된 공항기상관측장비에 의해 기상요소들이 매분 자동 관측된다. 특히 인천, 제주, 양양, 울산 등의 공항에서는 이·착륙 항공기에 영향을 미치는 저층난류를 탐측하기 위하여 저층난류경보장치를 운영하고 있다.

\subsection{고층기상관측}
고층기상관측은 지상보다 높은 상층 대기의 상태를 관측하는 것으로, 기상청은 레윈존데 관측, 수직측풍장비 관측을 수행한다. 레윈존데 관측은 기구에 라디오존데를 매달아 지상으로부터 약 35 km(5 hPa)까지의 고도별 기압, 기온, 습도, 풍향, 풍속을 00UTC와 12UTC에 관측하며, 수직측풍장비 관측은 UHF나 VHF 파장의 전파를 상층대기로 방사하고 바람과 함께 이동하는 난류에 산란되어 다시 수신되는 전파신호로 바람을 10분 간격으로 관측한다.

\subsection{해양기상관측}
해양기상관서, 해양기상관측부이, 해양기상영상감시시스템을 통하여 풍향·풍속, 기온, 수온, 기압, 파고 등을 관측하며, 먼바다의 기상현상 관측 및 부이 관리를 위하여 기상관측선을 운영하고 있다.

\subsection{기상위성관측}
기상위성은 우주공간에서 지구의 기상변화를 관측한다. 기상청은 정지궤도기상위성과 극궤도기상위성의 자료를 직접 수신하여 처리하여 예보를 위해 사용하고, 국민에게도 공개하고 있다. 이를 위해 기상위성 수신처리분석시스템을 서울, 문산, 서산에 설치하여 운영중이다.

\subsection{기상레이더 관측}
도플러 기상레이더를 설치하여 한반도에서 발생하는 악기상을 관측하여 예보에 활용하고 있다. 또한 일본 기상청과 공군의 레이더 자료도 수신하여 기존영상과 합성하여 종합적으로 활용하고 있다.


\section{자료수집}
통신용컴퓨터를 이용하여 국내기상자료와 외국에서 송신되는 각종기상자료를 수집, 편집, 가공하여 분석용 컴퓨터로 보낸다. 국내·외에서 수집된 관측자료로부터 수치예보모델을 이용하여 예상일기도를 생산한다. 이러한 수치예보모델의 운용을 위해 슈퍼컴퓨터가 사용된다.


%----------------------------------------------------------------------------------------
%	CHAPTER 
%----------------------------------------------------------------------------------------

\chapterimage{chapter_head_1.pdf} % Chapter heading image

\chapter{기상청(www.kma.go.kr) 일기 예보}

\section{기상청 일기 예보의 종류}


\subsection{초단기 동네 예보}
초단기예보는 현재부터 앞으로 3시간까지, 실황(날씨, 기온, 습도 등 7개 요소)과 예보(강수형태, 하늘상태, 강수량 등 3개 요소)를 1시간 간격으로 동네예보를 기반으로 매 시각 30분에 발표한다. 초단기예보는 짧은 시간에 발생·소멸하는 기상현상에 대해 신속하게 대응하여 재해예방에 최선을 다하고자 2010년 6월 15일부터 홈페이지를 통해 제공하고 있다.

\subsection{동네 예보}
대상기간과 구역을 시ㆍ공간적으로 세분화하여 발표하는 예보로 기온, 최고기온, 최저기온, 강수형태, 강수확률, 12시간강수량, 12시간적설, 하늘상태, 습도, 풍향, 풍속, 파고 등을 예보한다. 동네예보는 3시간 간격으로 1일에 8회 예보하며 예보구간도 역시 3시간 단위로 48시간까지 예보한다. 

\subsection{주간 예보}
기상전망, 예보구역별 육상 및 해상 날씨, 지점별 기온, 파고에 대한 48시간 이후부터  6일간의 예보로 일 2회 발표(06시, 18시)하는 주간예보(모레부터 6일간)가 계속 유지될 가능성에 대한 신뢰도 정보를 3단계로 구분하여 제공(육상)한다.

\begin{table}[h]
	\center
\begin{tabular}{c|c}
	\hline 
신뢰도  &	내용	  \\ 	\hline 
높음  & 다음날 발표 주간예보가 계속 유지될 가능성이 높음  \\  \hline 
보통 & 다음날 발표 주간예보가 계속 유지될 가능성이 있음  \\ 	\hline 
낮음 & 다음날 발표 주간예보가 계속 유지될 가능성이 낮음   \\ 	\hline 
\end{tabular} 
	\caption{신뢰도와 의미}
\end{table}

\subsection{주말 예보}
토요일과 일요일의 기상 개황, 일별 날씨, 야외활동 지수 등의 정보를 제공한다. 화요일 19시부터 금요일 24시까지 제공되며, 매일 19시에 발표한다.

\subsection{장기 예보}
장기예보는 11일 이상에 대한 예보를 일컬으며 순별·월별 기압계 동향 및 전망, 기온·강수량 예보 등을 발표한다. 예보구역은 한반도 12개 권역(서울·인천·경기도, 강원도 영서, 강원도 영동, 대전·충청남도, 충청북도, 광주·전라남도, 전라북도, 부산·울산·경상남도, 대구·경상북도, 제주도, 평안남북도·황해도, 함경남북도)이며, 월 3회 발표되는 1월 전망과 월 1회 발표되는 3개월 전망이 있다. 그 외 연 4회 발표되는 기후전망은 다음다음 계절에 대한 전망으로 봄철 기후전망은 11월 23일 경에, 여름철 기후전망은 2월 23일 경에, 가을철 기후전망은 5월 23일 경에 겨울철 기후전망은 8월 23일 경에 발표한다.

