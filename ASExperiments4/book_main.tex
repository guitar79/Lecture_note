%%%%%%%%%%%%%%%%%%%%%%%%%%%%%%%%%%%%%%%%%
% The Legrand Orange Book
% LaTeX Template
% Version 2.2 (30/3/17)
%
% This template has been downloaded from:
% http://www.LaTeXTemplates.com
%
% Original author:
% Mathias Legrand (legrand.mathias@gmail.com) with modifications by:
% Vel (vel@latextemplates.com)
%
% License:
% CC BY-NC-SA 3.0 (http://creativecommons.org/licenses/by-nc-sa/3.0/)
%
% Compiling this template:
% This template uses biber for its bibliography and makeindex for its index.
% When you first open the template, compile it from the command line with the 
% commands below to make sure your LaTeX distribution is configured correctly:
%
% 1) pdflatex main
% 2) makeindex main.idx -s StyleInd.ist
% 3) biber main
% 4) pdflatex main x 2
%
% After this, when you wish to update the bibliography/index use the appropriate
% command above and make sure to compile with pdflatex several times 
% afterwards to propagate your changes to the document.
%
% This template also uses a number of packages which may need to be
% updated to the newest versions for the template to compile. It is strongly
% recommended you update your LaTeX distribution if you have any
% compilation errors.
%
% Important note:
% Chapter heading images should have a 2:1 width:height ratio,
% e.g. 920px width and 460px height.
%
%%%%%%%%%%%%%%%%%%%%%%%%%%%%%%%%%%%%%%%%%

%----------------------------------------------------------------------------------------
%	PACKAGES AND OTHER DOCUMENT CONFIGURATIONS
%----------------------------------------------------------------------------------------

\documentclass[11pt,fleqn]{book} % Default font size and left-justified equations

%----------------------------------------------------------------------------------------

%%\documentclass{book_main}
%\usepackage{hangul} %<===> 유니코드/UTF-8 
\usepackage[hangul]{kotex} %<===> EUC-KR


\usepackage{ifxetex} % 부득이하게 pdflatex을 사용해도 문제가 없도록 함

\ifxetex
%한글 사용 옵션
\RequirePackage{xetexko}
\setmainfont[Ligatures=TeX]{Batang}
\setmainhangulfont[BoldFont=*,BoldFeatures=FakeBold,%
ItalicFont=*,ItalicFeatures=FakeSlant]{Batang}
\disablecjksymbolspacing
\nonfrenchspacing
\else
\fi

%%%%%%%%%%%%%%%%%%%%%%%%%%%%%%%%%%%%%%%%%
% The Legrand Orange Book
% Structural Definitions File
% Version 2.0 (9/2/15)
%
% Original author:
% Mathias Legrand (legrand.mathias@gmail.com) with modifications by:
% Vel (vel@latextemplates.com)
% 
% This file has been downloaded from:
% http://www.LaTeXTemplates.com
%
% License:
% CC BY-NC-SA 3.0 (http://creativecommons.org/licenses/by-nc-sa/3.0/)
%
%%%%%%%%%%%%%%%%%%%%%%%%%%%%%%%%%%%%%%%%%

%----------------------------------------------------------------------------------------
%	VARIOUS REQUIRED PACKAGES AND CONFIGURATIONS
%----------------------------------------------------------------------------------------

\usepackage[top=3cm,bottom=3cm,left=3cm,right=3cm,headsep=10pt,a4paper]{geometry} % Page margins

\usepackage{graphicx} % Required for including pictures
\graphicspath{{Pictures/}} % Specifies the directory where pictures are stored

\usepackage{lipsum} % Inserts dummy text

\usepackage{tikz} % Required for drawing custom shapes

\usepackage[english]{babel} % English language/hyphenation

\usepackage{enumitem} % Customize lists
\setlist{nolistsep} % Reduce spacing between bullet points and numbered lists

\usepackage{booktabs} % Required for nicer horizontal rules in tables

\usepackage{xcolor} % Required for specifying colors by name
\definecolor{ocre}{RGB}{243,102,25} % Define the orange color used for highlighting throughout the book

%----------------------------------------------------------------------------------------
%	FONTS
%----------------------------------------------------------------------------------------

\usepackage{avant} % Use the Avantgarde font for headings
%\usepackage{times} % Use the Times font for headings
\usepackage{mathptmx} % Use the Adobe Times Roman as the default text font together with math symbols from the Sym­bol, Chancery and Com­puter Modern fonts

\usepackage{microtype} % Slightly tweak font spacing for aesthetics
\usepackage[utf8]{inputenc} % Required for including letters with accents
\usepackage[T1]{fontenc} % Use 8-bit encoding that has 256 glyphs

%----------------------------------------------------------------------------------------
%	BIBLIOGRAPHY AND INDEX
%----------------------------------------------------------------------------------------

\usepackage[style=alphabetic,citestyle=numeric,sorting=nyt,sortcites=true,autopunct=true,babel=hyphen,hyperref=true,abbreviate=false,backref=true,backend=biber]{biblatex}
\addbibresource{bibliography.bib} % BibTeX bibliography file
\defbibheading{bibempty}{}

\usepackage{calc} % For simpler calculation - used for spacing the index letter headings correctly
\usepackage{makeidx} % Required to make an index
\makeindex % Tells LaTeX to create the files required for indexing

%----------------------------------------------------------------------------------------
%	MAIN TABLE OF CONTENTS
%----------------------------------------------------------------------------------------

\usepackage{titletoc} % Required for manipulating the table of contents

\contentsmargin{0cm} % Removes the default margin

% Part text styling
\titlecontents{part}[0cm]
{\addvspace{20pt}\centering\large\bfseries}
{}
{}
{}

% Chapter text styling
\titlecontents{chapter}[1.25cm] % Indentation
{\addvspace{12pt}\large\sffamily\bfseries} % Spacing and font options for chapters
{\color{ocre!60}\contentslabel[\Large\thecontentslabel]{1.25cm}\color{ocre}} % Chapter number
{\color{ocre}}
{\color{ocre!60}\normalsize\;\titlerule*[.5pc]{.}\;\thecontentspage} % Page number

% Section text styling
\titlecontents{section}[1.25cm] % Indentation
{\addvspace{3pt}\sffamily\bfseries} % Spacing and font options for sections
{\contentslabel[\thecontentslabel]{1.25cm}} % Section number
{}
{\hfill\color{black}\thecontentspage} % Page number
[]

% Subsection text styling
\titlecontents{subsection}[1.25cm] % Indentation
{\addvspace{1pt}\sffamily\small} % Spacing and font options for subsections
{\contentslabel[\thecontentslabel]{1.25cm}} % Subsection number
{}
{\ \titlerule*[.5pc]{.}\;\thecontentspage} % Page number
[]

% List of figures
\titlecontents{figure}[0em]
{\addvspace{-5pt}\sffamily}
{\thecontentslabel\hspace*{1em}}
{}
{\ \titlerule*[.5pc]{.}\;\thecontentspage}
[]

% List of tables
\titlecontents{table}[0em]
{\addvspace{-5pt}\sffamily}
{\thecontentslabel\hspace*{1em}}
{}
{\ \titlerule*[.5pc]{.}\;\thecontentspage}
[]

%----------------------------------------------------------------------------------------
%	MINI TABLE OF CONTENTS IN PART HEADS
%----------------------------------------------------------------------------------------

% Chapter text styling
\titlecontents{lchapter}[0em] % Indenting
{\addvspace{15pt}\large\sffamily\bfseries} % Spacing and font options for chapters
{\color{ocre}\contentslabel[\Large\thecontentslabel]{1.25cm}\color{ocre}} % Chapter number
{}  
{\color{ocre}\normalsize\sffamily\bfseries\;\titlerule*[.5pc]{.}\;\thecontentspage} % Page number

% Section text styling
\titlecontents{lsection}[0em] % Indenting
{\sffamily\small} % Spacing and font options for sections
{\contentslabel[\thecontentslabel]{1.25cm}} % Section number
{}
{}

% Subsection text styling
\titlecontents{lsubsection}[.5em] % Indentation
{\normalfont\footnotesize\sffamily} % Font settings
{}
{}
{}

%----------------------------------------------------------------------------------------
%	PAGE HEADERS
%----------------------------------------------------------------------------------------

\usepackage{fancyhdr} % Required for header and footer configuration

\pagestyle{fancy}
\renewcommand{\chaptermark}[1]{\markboth{\sffamily\normalsize\bfseries\chaptername\ \thechapter.\ #1}{}} % Chapter text font settings
\renewcommand{\sectionmark}[1]{\markright{\sffamily\normalsize\thesection\hspace{5pt}#1}{}} % Section text font settings
\fancyhf{} \fancyhead[LE,RO]{\sffamily\normalsize\thepage} % Font setting for the page number in the header
\fancyhead[LO]{\rightmark} % Print the nearest section name on the left side of odd pages
\fancyhead[RE]{\leftmark} % Print the current chapter name on the right side of even pages
\renewcommand{\headrulewidth}{0.5pt} % Width of the rule under the header
\addtolength{\headheight}{2.5pt} % Increase the spacing around the header slightly
\renewcommand{\footrulewidth}{0pt} % Removes the rule in the footer
\fancypagestyle{plain}{\fancyhead{}\renewcommand{\headrulewidth}{0pt}} % Style for when a plain pagestyle is specified

% Removes the header from odd empty pages at the end of chapters
\makeatletter
\renewcommand{\cleardoublepage}{
\clearpage\ifodd\c@page\else
\hbox{}
\vspace*{\fill}
\thispagestyle{empty}
\newpage
\fi}

%----------------------------------------------------------------------------------------
%	THEOREM STYLES
%----------------------------------------------------------------------------------------

\usepackage{amsmath,amsfonts,amssymb,amsthm} % For math equations, theorems, symbols, etc

\newcommand{\intoo}[2]{\mathopen{]}#1\,;#2\mathclose{[}}
\newcommand{\ud}{\mathop{\mathrm{{}d}}\mathopen{}}
\newcommand{\intff}[2]{\mathopen{[}#1\,;#2\mathclose{]}}
\newtheorem{notation}{Notation}[chapter]

% Boxed/framed environments
\newtheoremstyle{ocrenumbox}% % Theorem style name
{0pt}% Space above
{0pt}% Space below
{\normalfont}% % Body font
{}% Indent amount
{\small\bf\sffamily\color{ocre}}% % Theorem head font
{\;}% Punctuation after theorem head
{0.25em}% Space after theorem head
{\small\sffamily\color{ocre}\thmname{#1}\nobreakspace\thmnumber{\@ifnotempty{#1}{}\@upn{#2}}% Theorem text (e.g. Theorem 2.1)
\thmnote{\nobreakspace\the\thm@notefont\sffamily\bfseries\color{black}---\nobreakspace#3.}} % Optional theorem note
\renewcommand{\qedsymbol}{$\blacksquare$}% Optional qed square

\newtheoremstyle{blacknumex}% Theorem style name
{5pt}% Space above
{5pt}% Space below
{\normalfont}% Body font
{} % Indent amount
{\small\bf\sffamily}% Theorem head font
{\;}% Punctuation after theorem head
{0.25em}% Space after theorem head
{\small\sffamily{\tiny\ensuremath{\blacksquare}}\nobreakspace\thmname{#1}\nobreakspace\thmnumber{\@ifnotempty{#1}{}\@upn{#2}}% Theorem text (e.g. Theorem 2.1)
\thmnote{\nobreakspace\the\thm@notefont\sffamily\bfseries---\nobreakspace#3.}}% Optional theorem note

\newtheoremstyle{blacknumbox} % Theorem style name
{0pt}% Space above
{0pt}% Space below
{\normalfont}% Body font
{}% Indent amount
{\small\bf\sffamily}% Theorem head font
{\;}% Punctuation after theorem head
{0.25em}% Space after theorem head
{\small\sffamily\thmname{#1}\nobreakspace\thmnumber{\@ifnotempty{#1}{}\@upn{#2}}% Theorem text (e.g. Theorem 2.1)
\thmnote{\nobreakspace\the\thm@notefont\sffamily\bfseries---\nobreakspace#3.}}% Optional theorem note

% Non-boxed/non-framed environments
\newtheoremstyle{ocrenum}% % Theorem style name
{5pt}% Space above
{5pt}% Space below
{\normalfont}% % Body font
{}% Indent amount
{\small\bf\sffamily\color{ocre}}% % Theorem head font
{\;}% Punctuation after theorem head
{0.25em}% Space after theorem head
{\small\sffamily\color{ocre}\thmname{#1}\nobreakspace\thmnumber{\@ifnotempty{#1}{}\@upn{#2}}% Theorem text (e.g. Theorem 2.1)
\thmnote{\nobreakspace\the\thm@notefont\sffamily\bfseries\color{black}---\nobreakspace#3.}} % Optional theorem note
\renewcommand{\qedsymbol}{$\blacksquare$}% Optional qed square
\makeatother

% Defines the theorem text style for each type of theorem to one of the three styles above
\newcounter{dummy} 
\numberwithin{dummy}{section}
\theoremstyle{ocrenumbox}
\newtheorem{theoremeT}[dummy]{Theorem}
\newtheorem{problem}{Problem}[chapter]
\newtheorem{exerciseT}{Exercise}[chapter]
\theoremstyle{blacknumex}
\newtheorem{exampleT}{Example}[chapter]
\theoremstyle{blacknumbox}
\newtheorem{vocabulary}{Vocabulary}[chapter]
\newtheorem{definitionT}{Definition}[section]
\newtheorem{corollaryT}[dummy]{Corollary}
\theoremstyle{ocrenum}
\newtheorem{proposition}[dummy]{Proposition}

%----------------------------------------------------------------------------------------
%	DEFINITION OF COLORED BOXES
%----------------------------------------------------------------------------------------

\RequirePackage[framemethod=default]{mdframed} % Required for creating the theorem, definition, exercise and corollary boxes

% Theorem box
\newmdenv[skipabove=7pt,
skipbelow=7pt,
backgroundcolor=black!5,
linecolor=ocre,
innerleftmargin=5pt,
innerrightmargin=5pt,
innertopmargin=5pt,
leftmargin=0cm,
rightmargin=0cm,
innerbottommargin=5pt]{tBox}

% Exercise box	  
\newmdenv[skipabove=7pt,
skipbelow=7pt,
rightline=false,
leftline=true,
topline=false,
bottomline=false,
backgroundcolor=ocre!10,
linecolor=ocre,
innerleftmargin=5pt,
innerrightmargin=5pt,
innertopmargin=5pt,
innerbottommargin=5pt,
leftmargin=0cm,
rightmargin=0cm,
linewidth=4pt]{eBox}	

% Definition box
\newmdenv[skipabove=7pt,
skipbelow=7pt,
rightline=false,
leftline=true,
topline=false,
bottomline=false,
linecolor=ocre,
innerleftmargin=5pt,
innerrightmargin=5pt,
innertopmargin=0pt,
leftmargin=0cm,
rightmargin=0cm,
linewidth=4pt,
innerbottommargin=0pt]{dBox}	

% Corollary box
\newmdenv[skipabove=7pt,
skipbelow=7pt,
rightline=false,
leftline=true,
topline=false,
bottomline=false,
linecolor=gray,
backgroundcolor=black!5,
innerleftmargin=5pt,
innerrightmargin=5pt,
innertopmargin=5pt,
leftmargin=0cm,
rightmargin=0cm,
linewidth=4pt,
innerbottommargin=5pt]{cBox}

% Creates an environment for each type of theorem and assigns it a theorem text style from the "Theorem Styles" section above and a colored box from above
\newenvironment{theorem}{\begin{tBox}\begin{theoremeT}}{\end{theoremeT}\end{tBox}}
\newenvironment{exercise}{\begin{eBox}\begin{exerciseT}}{\hfill{\color{ocre}\tiny\ensuremath{\blacksquare}}\end{exerciseT}\end{eBox}}				  
\newenvironment{definition}{\begin{dBox}\begin{definitionT}}{\end{definitionT}\end{dBox}}	
\newenvironment{example}{\begin{exampleT}}{\hfill{\tiny\ensuremath{\blacksquare}}\end{exampleT}}		
\newenvironment{corollary}{\begin{cBox}\begin{corollaryT}}{\end{corollaryT}\end{cBox}}	

%----------------------------------------------------------------------------------------
%	REMARK ENVIRONMENT
%----------------------------------------------------------------------------------------

\newenvironment{remark}{\par\vspace{10pt}\small % Vertical white space above the remark and smaller font size
\begin{list}{}{
\leftmargin=35pt % Indentation on the left
\rightmargin=25pt}\item\ignorespaces % Indentation on the right
\makebox[-2.5pt]{\begin{tikzpicture}[overlay]
\node[draw=ocre!60,line width=1pt,circle,fill=ocre!25,font=\sffamily\bfseries,inner sep=2pt,outer sep=0pt] at (-15pt,0pt){\textcolor{ocre}{R}};\end{tikzpicture}} % Orange R in a circle
\advance\baselineskip -1pt}{\end{list}\vskip5pt} % Tighter line spacing and white space after remark

%----------------------------------------------------------------------------------------
%	SECTION NUMBERING IN THE MARGIN
%----------------------------------------------------------------------------------------

\makeatletter
\renewcommand{\@seccntformat}[1]{\llap{\textcolor{ocre}{\csname the#1\endcsname}\hspace{1em}}}                    
\renewcommand{\section}{\@startsection{section}{1}{\z@}
{-4ex \@plus -1ex \@minus -.4ex}
{1ex \@plus.2ex }
{\normalfont\large\sffamily\bfseries}}
\renewcommand{\subsection}{\@startsection {subsection}{2}{\z@}
{-3ex \@plus -0.1ex \@minus -.4ex}
{0.5ex \@plus.2ex }
{\normalfont\sffamily\bfseries}}
\renewcommand{\subsubsection}{\@startsection {subsubsection}{3}{\z@}
{-2ex \@plus -0.1ex \@minus -.2ex}
{.2ex \@plus.2ex }
{\normalfont\small\sffamily\bfseries}}                        
\renewcommand\paragraph{\@startsection{paragraph}{4}{\z@}
{-2ex \@plus-.2ex \@minus .2ex}
{.1ex}
{\normalfont\small\sffamily\bfseries}}

%----------------------------------------------------------------------------------------
%	PART HEADINGS
%----------------------------------------------------------------------------------------

% numbered part in the table of contents
\newcommand{\@mypartnumtocformat}[2]{%
\setlength\fboxsep{0pt}%
\noindent\colorbox{ocre!20}{\strut\parbox[c][.7cm]{\ecart}{\color{ocre!70}\Large\sffamily\bfseries\centering#1}}\hskip\esp\colorbox{ocre!40}{\strut\parbox[c][.7cm]{\linewidth-\ecart-\esp}{\Large\sffamily\centering#2}}}%
%%%%%%%%%%%%%%%%%%%%%%%%%%%%%%%%%%
% unnumbered part in the table of contents
\newcommand{\@myparttocformat}[1]{%
\setlength\fboxsep{0pt}%
\noindent\colorbox{ocre!40}{\strut\parbox[c][.7cm]{\linewidth}{\Large\sffamily\centering#1}}}%
%%%%%%%%%%%%%%%%%%%%%%%%%%%%%%%%%%
\newlength\esp
\setlength\esp{4pt}
\newlength\ecart
\setlength\ecart{1.2cm-\esp}
\newcommand{\thepartimage}{}%
\newcommand{\partimage}[1]{\renewcommand{\thepartimage}{#1}}%
\def\@part[#1]#2{%
\ifnum \c@secnumdepth >-2\relax%
\refstepcounter{part}%
\addcontentsline{toc}{part}{\texorpdfstring{\protect\@mypartnumtocformat{\thepart}{#1}}{\partname~\thepart\ ---\ #1}}
\else%
\addcontentsline{toc}{part}{\texorpdfstring{\protect\@myparttocformat{#1}}{#1}}%
\fi%
\startcontents%
\markboth{}{}%
{\thispagestyle{empty}%
\begin{tikzpicture}[remember picture,overlay]%
\node at (current page.north west){\begin{tikzpicture}[remember picture,overlay]%	
\fill[ocre!20](0cm,0cm) rectangle (\paperwidth,-\paperheight);
\node[anchor=north] at (4cm,-3.25cm){\color{ocre!40}\fontsize{220}{100}\sffamily\bfseries\thepart}; 
\node[anchor=south east] at (\paperwidth-1cm,-\paperheight+1cm){\parbox[t][][t]{8.5cm}{
\printcontents{l}{0}{\setcounter{tocdepth}{1}}%
}};
\node[anchor=north east] at (\paperwidth-1.5cm,-3.25cm){\parbox[t][][t]{15cm}{\strut\raggedleft\color{white}\fontsize{30}{30}\sffamily\bfseries#2}};
\end{tikzpicture}};
\end{tikzpicture}}%
\@endpart}
\def\@spart#1{%
\startcontents%
\phantomsection
{\thispagestyle{empty}%
\begin{tikzpicture}[remember picture,overlay]%
\node at (current page.north west){\begin{tikzpicture}[remember picture,overlay]%	
\fill[ocre!20](0cm,0cm) rectangle (\paperwidth,-\paperheight);
\node[anchor=north east] at (\paperwidth-1.5cm,-3.25cm){\parbox[t][][t]{15cm}{\strut\raggedleft\color{white}\fontsize{30}{30}\sffamily\bfseries#1}};
\end{tikzpicture}};
\end{tikzpicture}}
\addcontentsline{toc}{part}{\texorpdfstring{%
\setlength\fboxsep{0pt}%
\noindent\protect\colorbox{ocre!40}{\strut\protect\parbox[c][.7cm]{\linewidth}{\Large\sffamily\protect\centering #1\quad\mbox{}}}}{#1}}%
\@endpart}
\def\@endpart{\vfil\newpage
\if@twoside
\if@openright
\null
\thispagestyle{empty}%
\newpage
\fi
\fi
\if@tempswa
\twocolumn
\fi}

%----------------------------------------------------------------------------------------
%	CHAPTER HEADINGS
%----------------------------------------------------------------------------------------

% A switch to conditionally include a picture, implemented by  Christian Hupfer
\newif\ifusechapterimage
\usechapterimagetrue
\newcommand{\thechapterimage}{}%
\newcommand{\chapterimage}[1]{\ifusechapterimage\renewcommand{\thechapterimage}{#1}\fi}%
\newcommand{\autodot}{.}
\def\@makechapterhead#1{%
{\parindent \z@ \raggedright \normalfont
\ifnum \c@secnumdepth >\m@ne
\if@mainmatter
\begin{tikzpicture}[remember picture,overlay]
\node at (current page.north west)
{\begin{tikzpicture}[remember picture,overlay]
\node[anchor=north west,inner sep=0pt] at (0,0) {\ifusechapterimage\includegraphics[width=\paperwidth]{\thechapterimage}\fi};
\draw[anchor=west] (\Gm@lmargin,-9cm) node [line width=2pt,rounded corners=15pt,draw=ocre,fill=white,fill opacity=0.5,inner sep=15pt]{\strut\makebox[22cm]{}};
\draw[anchor=west] (\Gm@lmargin+.3cm,-9cm) node {\huge\sffamily\bfseries\color{black}\thechapter\autodot~#1\strut};
\end{tikzpicture}};
\end{tikzpicture}
\else
\begin{tikzpicture}[remember picture,overlay]
\node at (current page.north west)
{\begin{tikzpicture}[remember picture,overlay]
\node[anchor=north west,inner sep=0pt] at (0,0) {\ifusechapterimage\includegraphics[width=\paperwidth]{\thechapterimage}\fi};
\draw[anchor=west] (\Gm@lmargin,-9cm) node [line width=2pt,rounded corners=15pt,draw=ocre,fill=white,fill opacity=0.5,inner sep=15pt]{\strut\makebox[22cm]{}};
\draw[anchor=west] (\Gm@lmargin+.3cm,-9cm) node {\huge\sffamily\bfseries\color{black}#1\strut};
\end{tikzpicture}};
\end{tikzpicture}
\fi\fi\par\vspace*{270\p@}}}

%-------------------------------------------

\def\@makeschapterhead#1{%
\begin{tikzpicture}[remember picture,overlay]
\node at (current page.north west)
{\begin{tikzpicture}[remember picture,overlay]
\node[anchor=north west,inner sep=0pt] at (0,0) {\ifusechapterimage\includegraphics[width=\paperwidth]{\thechapterimage}\fi};
\draw[anchor=west] (\Gm@lmargin,-9cm) node [line width=2pt,rounded corners=15pt,draw=ocre,fill=white,fill opacity=0.5,inner sep=15pt]{\strut\makebox[22cm]{}};
\draw[anchor=west] (\Gm@lmargin+.3cm,-9cm) node {\huge\sffamily\bfseries\color{black}#1\strut};
\end{tikzpicture}};
\end{tikzpicture}
\par\vspace*{270\p@}}
\makeatother

%----------------------------------------------------------------------------------------
%	HYPERLINKS IN THE DOCUMENTS
%----------------------------------------------------------------------------------------

\usepackage{hyperref}
\hypersetup{hidelinks,backref=true,pagebackref=true,hyperindex=true,colorlinks=false,breaklinks=true,urlcolor= ocre,bookmarks=true,bookmarksopen=false,pdftitle={Title},pdfauthor={Author}}
\usepackage{bookmark}
\bookmarksetup{
open,
numbered,
addtohook={%
\ifnum\bookmarkget{level}=0 % chapter
\bookmarksetup{bold}%
\fi
\ifnum\bookmarkget{level}=-1 % part
\bookmarksetup{color=ocre,bold}%
\fi
}
}
 % Insert the commands.tex file which contains the majority of the structure behind the template
%%\documentclass{book_main}
%\usepackage{hangul} %<===> 유니코드/UTF-8 
\usepackage[hangul]{kotex} %<===> EUC-KR


\usepackage{ifxetex} % 부득이하게 pdflatex을 사용해도 문제가 없도록 함

\ifxetex
%한글 사용 옵션
\RequirePackage{xetexko}
\setmainfont[Ligatures=TeX]{Batang}
\setmainhangulfont[BoldFont=*,BoldFeatures=FakeBold,%
ItalicFont=*,ItalicFeatures=FakeSlant]{Batang}
\disablecjksymbolspacing
\nonfrenchspacing
\else
\fi

%%%%%%%%%%%%%%%%%%%%%%%%%%%%%%%%%%%%%%%%%
% The Legrand Orange Book
% Structural Definitions File
% Version 2.0 (9/2/15)
%
% Original author:
% Mathias Legrand (legrand.mathias@gmail.com) with modifications by:
% Vel (vel@latextemplates.com)
% 
% This file has been downloaded from:
% http://www.LaTeXTemplates.com
%
% License:
% CC BY-NC-SA 3.0 (http://creativecommons.org/licenses/by-nc-sa/3.0/)
%
%%%%%%%%%%%%%%%%%%%%%%%%%%%%%%%%%%%%%%%%%

%----------------------------------------------------------------------------------------
%	VARIOUS REQUIRED PACKAGES AND CONFIGURATIONS
%----------------------------------------------------------------------------------------

\usepackage[top=3cm,bottom=3cm,left=3cm,right=3cm,headsep=10pt,a4paper]{geometry} % Page margins

\usepackage{graphicx} % Required for including pictures
\graphicspath{{Pictures/}} % Specifies the directory where pictures are stored

\usepackage{lipsum} % Inserts dummy text

\usepackage{tikz} % Required for drawing custom shapes

\usepackage[english]{babel} % English language/hyphenation

\usepackage{enumitem} % Customize lists
\setlist{nolistsep} % Reduce spacing between bullet points and numbered lists

\usepackage{booktabs} % Required for nicer horizontal rules in tables

\usepackage{xcolor} % Required for specifying colors by name
\definecolor{ocre}{RGB}{243,102,25} % Define the orange color used for highlighting throughout the book

%----------------------------------------------------------------------------------------
%	FONTS
%----------------------------------------------------------------------------------------

\usepackage{avant} % Use the Avantgarde font for headings
%\usepackage{times} % Use the Times font for headings
\usepackage{mathptmx} % Use the Adobe Times Roman as the default text font together with math symbols from the Sym­bol, Chancery and Com­puter Modern fonts

\usepackage{microtype} % Slightly tweak font spacing for aesthetics
\usepackage[utf8]{inputenc} % Required for including letters with accents
\usepackage[T1]{fontenc} % Use 8-bit encoding that has 256 glyphs

%----------------------------------------------------------------------------------------
%	BIBLIOGRAPHY AND INDEX
%----------------------------------------------------------------------------------------

\usepackage[style=alphabetic,citestyle=numeric,sorting=nyt,sortcites=true,autopunct=true,babel=hyphen,hyperref=true,abbreviate=false,backref=true,backend=biber]{biblatex}
\addbibresource{bibliography.bib} % BibTeX bibliography file
\defbibheading{bibempty}{}

\usepackage{calc} % For simpler calculation - used for spacing the index letter headings correctly
\usepackage{makeidx} % Required to make an index
\makeindex % Tells LaTeX to create the files required for indexing

%----------------------------------------------------------------------------------------
%	MAIN TABLE OF CONTENTS
%----------------------------------------------------------------------------------------

\usepackage{titletoc} % Required for manipulating the table of contents

\contentsmargin{0cm} % Removes the default margin

% Part text styling
\titlecontents{part}[0cm]
{\addvspace{20pt}\centering\large\bfseries}
{}
{}
{}

% Chapter text styling
\titlecontents{chapter}[1.25cm] % Indentation
{\addvspace{12pt}\large\sffamily\bfseries} % Spacing and font options for chapters
{\color{ocre!60}\contentslabel[\Large\thecontentslabel]{1.25cm}\color{ocre}} % Chapter number
{\color{ocre}}
{\color{ocre!60}\normalsize\;\titlerule*[.5pc]{.}\;\thecontentspage} % Page number

% Section text styling
\titlecontents{section}[1.25cm] % Indentation
{\addvspace{3pt}\sffamily\bfseries} % Spacing and font options for sections
{\contentslabel[\thecontentslabel]{1.25cm}} % Section number
{}
{\hfill\color{black}\thecontentspage} % Page number
[]

% Subsection text styling
\titlecontents{subsection}[1.25cm] % Indentation
{\addvspace{1pt}\sffamily\small} % Spacing and font options for subsections
{\contentslabel[\thecontentslabel]{1.25cm}} % Subsection number
{}
{\ \titlerule*[.5pc]{.}\;\thecontentspage} % Page number
[]

% List of figures
\titlecontents{figure}[0em]
{\addvspace{-5pt}\sffamily}
{\thecontentslabel\hspace*{1em}}
{}
{\ \titlerule*[.5pc]{.}\;\thecontentspage}
[]

% List of tables
\titlecontents{table}[0em]
{\addvspace{-5pt}\sffamily}
{\thecontentslabel\hspace*{1em}}
{}
{\ \titlerule*[.5pc]{.}\;\thecontentspage}
[]

%----------------------------------------------------------------------------------------
%	MINI TABLE OF CONTENTS IN PART HEADS
%----------------------------------------------------------------------------------------

% Chapter text styling
\titlecontents{lchapter}[0em] % Indenting
{\addvspace{15pt}\large\sffamily\bfseries} % Spacing and font options for chapters
{\color{ocre}\contentslabel[\Large\thecontentslabel]{1.25cm}\color{ocre}} % Chapter number
{}  
{\color{ocre}\normalsize\sffamily\bfseries\;\titlerule*[.5pc]{.}\;\thecontentspage} % Page number

% Section text styling
\titlecontents{lsection}[0em] % Indenting
{\sffamily\small} % Spacing and font options for sections
{\contentslabel[\thecontentslabel]{1.25cm}} % Section number
{}
{}

% Subsection text styling
\titlecontents{lsubsection}[.5em] % Indentation
{\normalfont\footnotesize\sffamily} % Font settings
{}
{}
{}

%----------------------------------------------------------------------------------------
%	PAGE HEADERS
%----------------------------------------------------------------------------------------

\usepackage{fancyhdr} % Required for header and footer configuration

\pagestyle{fancy}
\renewcommand{\chaptermark}[1]{\markboth{\sffamily\normalsize\bfseries\chaptername\ \thechapter.\ #1}{}} % Chapter text font settings
\renewcommand{\sectionmark}[1]{\markright{\sffamily\normalsize\thesection\hspace{5pt}#1}{}} % Section text font settings
\fancyhf{} \fancyhead[LE,RO]{\sffamily\normalsize\thepage} % Font setting for the page number in the header
\fancyhead[LO]{\rightmark} % Print the nearest section name on the left side of odd pages
\fancyhead[RE]{\leftmark} % Print the current chapter name on the right side of even pages
\renewcommand{\headrulewidth}{0.5pt} % Width of the rule under the header
\addtolength{\headheight}{2.5pt} % Increase the spacing around the header slightly
\renewcommand{\footrulewidth}{0pt} % Removes the rule in the footer
\fancypagestyle{plain}{\fancyhead{}\renewcommand{\headrulewidth}{0pt}} % Style for when a plain pagestyle is specified

% Removes the header from odd empty pages at the end of chapters
\makeatletter
\renewcommand{\cleardoublepage}{
\clearpage\ifodd\c@page\else
\hbox{}
\vspace*{\fill}
\thispagestyle{empty}
\newpage
\fi}

%----------------------------------------------------------------------------------------
%	THEOREM STYLES
%----------------------------------------------------------------------------------------

\usepackage{amsmath,amsfonts,amssymb,amsthm} % For math equations, theorems, symbols, etc

\newcommand{\intoo}[2]{\mathopen{]}#1\,;#2\mathclose{[}}
\newcommand{\ud}{\mathop{\mathrm{{}d}}\mathopen{}}
\newcommand{\intff}[2]{\mathopen{[}#1\,;#2\mathclose{]}}
\newtheorem{notation}{Notation}[chapter]

% Boxed/framed environments
\newtheoremstyle{ocrenumbox}% % Theorem style name
{0pt}% Space above
{0pt}% Space below
{\normalfont}% % Body font
{}% Indent amount
{\small\bf\sffamily\color{ocre}}% % Theorem head font
{\;}% Punctuation after theorem head
{0.25em}% Space after theorem head
{\small\sffamily\color{ocre}\thmname{#1}\nobreakspace\thmnumber{\@ifnotempty{#1}{}\@upn{#2}}% Theorem text (e.g. Theorem 2.1)
\thmnote{\nobreakspace\the\thm@notefont\sffamily\bfseries\color{black}---\nobreakspace#3.}} % Optional theorem note
\renewcommand{\qedsymbol}{$\blacksquare$}% Optional qed square

\newtheoremstyle{blacknumex}% Theorem style name
{5pt}% Space above
{5pt}% Space below
{\normalfont}% Body font
{} % Indent amount
{\small\bf\sffamily}% Theorem head font
{\;}% Punctuation after theorem head
{0.25em}% Space after theorem head
{\small\sffamily{\tiny\ensuremath{\blacksquare}}\nobreakspace\thmname{#1}\nobreakspace\thmnumber{\@ifnotempty{#1}{}\@upn{#2}}% Theorem text (e.g. Theorem 2.1)
\thmnote{\nobreakspace\the\thm@notefont\sffamily\bfseries---\nobreakspace#3.}}% Optional theorem note

\newtheoremstyle{blacknumbox} % Theorem style name
{0pt}% Space above
{0pt}% Space below
{\normalfont}% Body font
{}% Indent amount
{\small\bf\sffamily}% Theorem head font
{\;}% Punctuation after theorem head
{0.25em}% Space after theorem head
{\small\sffamily\thmname{#1}\nobreakspace\thmnumber{\@ifnotempty{#1}{}\@upn{#2}}% Theorem text (e.g. Theorem 2.1)
\thmnote{\nobreakspace\the\thm@notefont\sffamily\bfseries---\nobreakspace#3.}}% Optional theorem note

% Non-boxed/non-framed environments
\newtheoremstyle{ocrenum}% % Theorem style name
{5pt}% Space above
{5pt}% Space below
{\normalfont}% % Body font
{}% Indent amount
{\small\bf\sffamily\color{ocre}}% % Theorem head font
{\;}% Punctuation after theorem head
{0.25em}% Space after theorem head
{\small\sffamily\color{ocre}\thmname{#1}\nobreakspace\thmnumber{\@ifnotempty{#1}{}\@upn{#2}}% Theorem text (e.g. Theorem 2.1)
\thmnote{\nobreakspace\the\thm@notefont\sffamily\bfseries\color{black}---\nobreakspace#3.}} % Optional theorem note
\renewcommand{\qedsymbol}{$\blacksquare$}% Optional qed square
\makeatother

% Defines the theorem text style for each type of theorem to one of the three styles above
\newcounter{dummy} 
\numberwithin{dummy}{section}
\theoremstyle{ocrenumbox}
\newtheorem{theoremeT}[dummy]{Theorem}
\newtheorem{problem}{Problem}[chapter]
\newtheorem{exerciseT}{Exercise}[chapter]
\theoremstyle{blacknumex}
\newtheorem{exampleT}{Example}[chapter]
\theoremstyle{blacknumbox}
\newtheorem{vocabulary}{Vocabulary}[chapter]
\newtheorem{definitionT}{Definition}[section]
\newtheorem{corollaryT}[dummy]{Corollary}
\theoremstyle{ocrenum}
\newtheorem{proposition}[dummy]{Proposition}

%----------------------------------------------------------------------------------------
%	DEFINITION OF COLORED BOXES
%----------------------------------------------------------------------------------------

\RequirePackage[framemethod=default]{mdframed} % Required for creating the theorem, definition, exercise and corollary boxes

% Theorem box
\newmdenv[skipabove=7pt,
skipbelow=7pt,
backgroundcolor=black!5,
linecolor=ocre,
innerleftmargin=5pt,
innerrightmargin=5pt,
innertopmargin=5pt,
leftmargin=0cm,
rightmargin=0cm,
innerbottommargin=5pt]{tBox}

% Exercise box	  
\newmdenv[skipabove=7pt,
skipbelow=7pt,
rightline=false,
leftline=true,
topline=false,
bottomline=false,
backgroundcolor=ocre!10,
linecolor=ocre,
innerleftmargin=5pt,
innerrightmargin=5pt,
innertopmargin=5pt,
innerbottommargin=5pt,
leftmargin=0cm,
rightmargin=0cm,
linewidth=4pt]{eBox}	

% Definition box
\newmdenv[skipabove=7pt,
skipbelow=7pt,
rightline=false,
leftline=true,
topline=false,
bottomline=false,
linecolor=ocre,
innerleftmargin=5pt,
innerrightmargin=5pt,
innertopmargin=0pt,
leftmargin=0cm,
rightmargin=0cm,
linewidth=4pt,
innerbottommargin=0pt]{dBox}	

% Corollary box
\newmdenv[skipabove=7pt,
skipbelow=7pt,
rightline=false,
leftline=true,
topline=false,
bottomline=false,
linecolor=gray,
backgroundcolor=black!5,
innerleftmargin=5pt,
innerrightmargin=5pt,
innertopmargin=5pt,
leftmargin=0cm,
rightmargin=0cm,
linewidth=4pt,
innerbottommargin=5pt]{cBox}

% Creates an environment for each type of theorem and assigns it a theorem text style from the "Theorem Styles" section above and a colored box from above
\newenvironment{theorem}{\begin{tBox}\begin{theoremeT}}{\end{theoremeT}\end{tBox}}
\newenvironment{exercise}{\begin{eBox}\begin{exerciseT}}{\hfill{\color{ocre}\tiny\ensuremath{\blacksquare}}\end{exerciseT}\end{eBox}}				  
\newenvironment{definition}{\begin{dBox}\begin{definitionT}}{\end{definitionT}\end{dBox}}	
\newenvironment{example}{\begin{exampleT}}{\hfill{\tiny\ensuremath{\blacksquare}}\end{exampleT}}		
\newenvironment{corollary}{\begin{cBox}\begin{corollaryT}}{\end{corollaryT}\end{cBox}}	

%----------------------------------------------------------------------------------------
%	REMARK ENVIRONMENT
%----------------------------------------------------------------------------------------

\newenvironment{remark}{\par\vspace{10pt}\small % Vertical white space above the remark and smaller font size
\begin{list}{}{
\leftmargin=35pt % Indentation on the left
\rightmargin=25pt}\item\ignorespaces % Indentation on the right
\makebox[-2.5pt]{\begin{tikzpicture}[overlay]
\node[draw=ocre!60,line width=1pt,circle,fill=ocre!25,font=\sffamily\bfseries,inner sep=2pt,outer sep=0pt] at (-15pt,0pt){\textcolor{ocre}{R}};\end{tikzpicture}} % Orange R in a circle
\advance\baselineskip -1pt}{\end{list}\vskip5pt} % Tighter line spacing and white space after remark

%----------------------------------------------------------------------------------------
%	SECTION NUMBERING IN THE MARGIN
%----------------------------------------------------------------------------------------

\makeatletter
\renewcommand{\@seccntformat}[1]{\llap{\textcolor{ocre}{\csname the#1\endcsname}\hspace{1em}}}                    
\renewcommand{\section}{\@startsection{section}{1}{\z@}
{-4ex \@plus -1ex \@minus -.4ex}
{1ex \@plus.2ex }
{\normalfont\large\sffamily\bfseries}}
\renewcommand{\subsection}{\@startsection {subsection}{2}{\z@}
{-3ex \@plus -0.1ex \@minus -.4ex}
{0.5ex \@plus.2ex }
{\normalfont\sffamily\bfseries}}
\renewcommand{\subsubsection}{\@startsection {subsubsection}{3}{\z@}
{-2ex \@plus -0.1ex \@minus -.2ex}
{.2ex \@plus.2ex }
{\normalfont\small\sffamily\bfseries}}                        
\renewcommand\paragraph{\@startsection{paragraph}{4}{\z@}
{-2ex \@plus-.2ex \@minus .2ex}
{.1ex}
{\normalfont\small\sffamily\bfseries}}

%----------------------------------------------------------------------------------------
%	PART HEADINGS
%----------------------------------------------------------------------------------------

% numbered part in the table of contents
\newcommand{\@mypartnumtocformat}[2]{%
\setlength\fboxsep{0pt}%
\noindent\colorbox{ocre!20}{\strut\parbox[c][.7cm]{\ecart}{\color{ocre!70}\Large\sffamily\bfseries\centering#1}}\hskip\esp\colorbox{ocre!40}{\strut\parbox[c][.7cm]{\linewidth-\ecart-\esp}{\Large\sffamily\centering#2}}}%
%%%%%%%%%%%%%%%%%%%%%%%%%%%%%%%%%%
% unnumbered part in the table of contents
\newcommand{\@myparttocformat}[1]{%
\setlength\fboxsep{0pt}%
\noindent\colorbox{ocre!40}{\strut\parbox[c][.7cm]{\linewidth}{\Large\sffamily\centering#1}}}%
%%%%%%%%%%%%%%%%%%%%%%%%%%%%%%%%%%
\newlength\esp
\setlength\esp{4pt}
\newlength\ecart
\setlength\ecart{1.2cm-\esp}
\newcommand{\thepartimage}{}%
\newcommand{\partimage}[1]{\renewcommand{\thepartimage}{#1}}%
\def\@part[#1]#2{%
\ifnum \c@secnumdepth >-2\relax%
\refstepcounter{part}%
\addcontentsline{toc}{part}{\texorpdfstring{\protect\@mypartnumtocformat{\thepart}{#1}}{\partname~\thepart\ ---\ #1}}
\else%
\addcontentsline{toc}{part}{\texorpdfstring{\protect\@myparttocformat{#1}}{#1}}%
\fi%
\startcontents%
\markboth{}{}%
{\thispagestyle{empty}%
\begin{tikzpicture}[remember picture,overlay]%
\node at (current page.north west){\begin{tikzpicture}[remember picture,overlay]%	
\fill[ocre!20](0cm,0cm) rectangle (\paperwidth,-\paperheight);
\node[anchor=north] at (4cm,-3.25cm){\color{ocre!40}\fontsize{220}{100}\sffamily\bfseries\thepart}; 
\node[anchor=south east] at (\paperwidth-1cm,-\paperheight+1cm){\parbox[t][][t]{8.5cm}{
\printcontents{l}{0}{\setcounter{tocdepth}{1}}%
}};
\node[anchor=north east] at (\paperwidth-1.5cm,-3.25cm){\parbox[t][][t]{15cm}{\strut\raggedleft\color{white}\fontsize{30}{30}\sffamily\bfseries#2}};
\end{tikzpicture}};
\end{tikzpicture}}%
\@endpart}
\def\@spart#1{%
\startcontents%
\phantomsection
{\thispagestyle{empty}%
\begin{tikzpicture}[remember picture,overlay]%
\node at (current page.north west){\begin{tikzpicture}[remember picture,overlay]%	
\fill[ocre!20](0cm,0cm) rectangle (\paperwidth,-\paperheight);
\node[anchor=north east] at (\paperwidth-1.5cm,-3.25cm){\parbox[t][][t]{15cm}{\strut\raggedleft\color{white}\fontsize{30}{30}\sffamily\bfseries#1}};
\end{tikzpicture}};
\end{tikzpicture}}
\addcontentsline{toc}{part}{\texorpdfstring{%
\setlength\fboxsep{0pt}%
\noindent\protect\colorbox{ocre!40}{\strut\protect\parbox[c][.7cm]{\linewidth}{\Large\sffamily\protect\centering #1\quad\mbox{}}}}{#1}}%
\@endpart}
\def\@endpart{\vfil\newpage
\if@twoside
\if@openright
\null
\thispagestyle{empty}%
\newpage
\fi
\fi
\if@tempswa
\twocolumn
\fi}

%----------------------------------------------------------------------------------------
%	CHAPTER HEADINGS
%----------------------------------------------------------------------------------------

% A switch to conditionally include a picture, implemented by  Christian Hupfer
\newif\ifusechapterimage
\usechapterimagetrue
\newcommand{\thechapterimage}{}%
\newcommand{\chapterimage}[1]{\ifusechapterimage\renewcommand{\thechapterimage}{#1}\fi}%
\newcommand{\autodot}{.}
\def\@makechapterhead#1{%
{\parindent \z@ \raggedright \normalfont
\ifnum \c@secnumdepth >\m@ne
\if@mainmatter
\begin{tikzpicture}[remember picture,overlay]
\node at (current page.north west)
{\begin{tikzpicture}[remember picture,overlay]
\node[anchor=north west,inner sep=0pt] at (0,0) {\ifusechapterimage\includegraphics[width=\paperwidth]{\thechapterimage}\fi};
\draw[anchor=west] (\Gm@lmargin,-9cm) node [line width=2pt,rounded corners=15pt,draw=ocre,fill=white,fill opacity=0.5,inner sep=15pt]{\strut\makebox[22cm]{}};
\draw[anchor=west] (\Gm@lmargin+.3cm,-9cm) node {\huge\sffamily\bfseries\color{black}\thechapter\autodot~#1\strut};
\end{tikzpicture}};
\end{tikzpicture}
\else
\begin{tikzpicture}[remember picture,overlay]
\node at (current page.north west)
{\begin{tikzpicture}[remember picture,overlay]
\node[anchor=north west,inner sep=0pt] at (0,0) {\ifusechapterimage\includegraphics[width=\paperwidth]{\thechapterimage}\fi};
\draw[anchor=west] (\Gm@lmargin,-9cm) node [line width=2pt,rounded corners=15pt,draw=ocre,fill=white,fill opacity=0.5,inner sep=15pt]{\strut\makebox[22cm]{}};
\draw[anchor=west] (\Gm@lmargin+.3cm,-9cm) node {\huge\sffamily\bfseries\color{black}#1\strut};
\end{tikzpicture}};
\end{tikzpicture}
\fi\fi\par\vspace*{270\p@}}}

%-------------------------------------------

\def\@makeschapterhead#1{%
\begin{tikzpicture}[remember picture,overlay]
\node at (current page.north west)
{\begin{tikzpicture}[remember picture,overlay]
\node[anchor=north west,inner sep=0pt] at (0,0) {\ifusechapterimage\includegraphics[width=\paperwidth]{\thechapterimage}\fi};
\draw[anchor=west] (\Gm@lmargin,-9cm) node [line width=2pt,rounded corners=15pt,draw=ocre,fill=white,fill opacity=0.5,inner sep=15pt]{\strut\makebox[22cm]{}};
\draw[anchor=west] (\Gm@lmargin+.3cm,-9cm) node {\huge\sffamily\bfseries\color{black}#1\strut};
\end{tikzpicture}};
\end{tikzpicture}
\par\vspace*{270\p@}}
\makeatother

%----------------------------------------------------------------------------------------
%	HYPERLINKS IN THE DOCUMENTS
%----------------------------------------------------------------------------------------

\usepackage{hyperref}
\hypersetup{hidelinks,backref=true,pagebackref=true,hyperindex=true,colorlinks=false,breaklinks=true,urlcolor= ocre,bookmarks=true,bookmarksopen=false,pdftitle={Title},pdfauthor={Author}}
\usepackage{bookmark}
\bookmarksetup{
open,
numbered,
addtohook={%
\ifnum\bookmarkget{level}=0 % chapter
\bookmarksetup{bold}%
\fi
\ifnum\bookmarkget{level}=-1 % part
\bookmarksetup{color=ocre,bold}%
\fi
}
}
 % Insert the commands.tex file which contains the majority of the
\documentclass{gshs_thesis}

\graphicspath{{images/}}
% 이곳에 필요한 별도의 패키지들을 적어넣으시오.
%\usepackage{...}
\usepackage{verbatim} % for commment, verbatim environment
\usepackage{spverbatim} % automatic linebreak verbatim environment
\usepackage{listings}
\lstset{
	basicstyle=\small\ttfamily,
	columns=flexible,
	breaklines=true
}
\usepackage{hologo}

% -----------------------------------------------------------------------
%                   이 부분은 수정하지 마시오.
% -----------------------------------------------------------------------
\titleheader{졸업논문청구논문}
\school{과학영재학교 경기과학고등학교}
\approval{위 논문은 과학영재학교 경기과학고등학교 졸업논문으로\\
졸업논문심사위원회에서 심사 통과하였음.}
\chairperson{심사위원장}
\examiner{심사위원}
\apprvsign{(인)}
\korabstract{초 록}
\koracknowledgement{감사의 글}
\korresearches{연 구 활 동}

%: ----------------------------------------------------------------------
%:                  논문 제목과 저자 이름을 입력하시오
% ----------------------------------------------------------------------
\title{한글 제목} %한글 제목
\engtitle{English Title} %영문 제목
\korname{홍 길 동} %저자 이름을 한글로 입력하시오 (글자 사이 띄어쓰기)
\engname{Hong, Gil-Dong} %저자 이름을 영어로 입력하시오 (family name, personal name)
\chnname{洪 吉 東} %저자 이름을 한자로 입력하시오 (글자 사이 띄어쓰기)
\studid{14201} %학번을 입력하시오

%------------------------------------------------------------------------
%                  심사위원과 논문 승인 날짜를 입력하시오
%------------------------------------------------------------------------
\advisor{Mok, Chinook}  %지도교사 영문 이름 (family name, personal name)
\judgeone{박 승 원} %심사위원장
\judgetwo{이 주 찬}   %심사위원1
\judgethree{목 진 욱} %심사위원2(지도교사)
\degreeyear{2017}   %졸업 년도
\degreedate{2016}{11}{13} %논문 승인 날짜 양식

%------------------------------------------------------------------------
%                  논문제출 전 체크리스트를 확인하시오
%------------------------------------------------------------------------
\checklisttitle{[논문제출 전 체크리스트]} %수정하지 마시오
\checklistI{1. 이 논문은 내가 직접 연구하고 작성한 것이다.} %수정하지 마시오
% 이 항목이 사실이라면 다음 줄 앞에 "%"기호 삽입, 다다음 줄 앞의 "%"기호 제거하시오
\checklistmarkI{$\square$}
%\checklistmarkI{$\text{\rlap{$\checkmark$}}\square$}
\checklistII{2. 인용한 모든 자료(책, 논문, 인터넷자료 등)의 인용표시를 바르게 하였다.} %수정하지 마시오
% 이 항목이 사실이라면 다음 줄 앞에 "%"기호 삽입, 다다음 줄 앞의 "%"기호 제거하시오
\checklistmarkII{$\square$}
%\checklistmarkII{$\text{\rlap{$\checkmark$}}\square$}
\checklistIII{3. 인용한 자료의 표현이나 내용을 왜곡하지 않았다.} %수정하지마시오
% 이 항목이 사실이라면 다음 줄 앞에 "%"기호 삽입, 다다음 줄 앞의 "%"기호 제거하시오
\checklistmarkIII{$\square$}
%\checklistmarkIII{$\text{\rlap{$\checkmark$}}\square$}
\checklistIV{4. 정확한 출처제시 없이 다른 사람의 글이나 아이디어를 가져오지 않았다.} %수정하지 마시오
% 이 항목이 사실이라면 다음 줄 앞에 "%"기호 삽입, 다다음 줄 앞의 "%"기호 제거하시오
\checklistmarkIV{$\square$}
%\checklistmarkIV{$\text{\rlap{$\checkmark$}}\square$}
\checklistV{5. 논문 작성 중 도표나 데이터를 조작(위조 혹은 변조)하지 않았다.} %수정하지 마시오
% 이 항목이 사실이라면 다음 줄 앞에 "%"기호 삽입, 다다음 줄 앞의 "%"기호 제거하시오
\checklistmarkV{$\square$}
%\checklistmarkV{$\text{\rlap{$\checkmark$}}\square$}
\checklistVI{6. 다른 친구와 같은 내용의 논문을 제출하지 않았다.} %수정하지 마시오
% 이 항목이 사실이라면 다음 줄 앞에 "%"기호 삽입, 다다음 줄 앞의 "%"기호 제거하시오
\checklistmarkVI{$\square$}
%\checklistmarkVI{$\text{\rlap{$\checkmark$}}\square$} % Insert the commands.tex file which contains the majority of the structure behind the template
\begin{document}


%----------------------------------------------------------------------------------------
%	TITLE PAGE
%----------------------------------------------------------------------------------------

\begingroup
\thispagestyle{empty}
\begin{tikzpicture}[remember picture,overlay]
\node[inner sep=0pt] (background) at (current page.center) {\includegraphics[width=\paperwidth]{background}};
\draw (current page.center) node [fill=ocre!30!white,fill opacity=0.6,text opacity=1,inner sep=1cm]{\Huge\centering\bfseries\sffamily\parbox[c][][t]{\paperwidth}{\centering 대기과학 실험\\[15pt] % Book title
		{\Large Atmospheric science experiments}\\[20pt] % Subtitle
		{\huge 박 기 현 }}}; % Author name
\end{tikzpicture}
\vfill
\endgroup

%----------------------------------------------------------------------------------------
%	COPYRIGHT PAGE
%----------------------------------------------------------------------------------------

\newpage
~\vfill
\thispagestyle{empty}

\noindent Copyright \copyright\ 2017 Park, Kie-hyun\\ % Copyright notice

\noindent \textsc{Published by 경기과학고등학교}\\ % Publisher

\noindent \textsc{www.gs.hs.kr}\\ % URL

\noindent Licensed under the Creative Commons Attribution-NonCommercial 3.0 Unported License (the ``License''). You may not use this file except in compliance with the License. You may obtain a copy of the License at \url{http://creativecommons.org/licenses/by-nc/3.0}. Unless required by applicable law or agreed to in writing, software distributed under the License is distributed on an \textsc{``as is'' basis, without warranties or conditions of any kind}, either express or implied. See the License for the specific language governing permissions and limitations under the License.\\ % License information

\noindent \textit{First printing, August 2017} % Printing/edition date

%----------------------------------------------------------------------------------------
%	TABLE OF CONTENTS
%----------------------------------------------------------------------------------------

%\usechapterimagefalse % If you don't want to include a chapter image, use this to toggle images off - it can be enabled later with \usechapterimagetrue

\chapterimage{chapter_head_1.pdf} % Table of contents heading image

\pagestyle{empty} % No headers

\tableofcontents % Print the table of contents itself

\cleardoublepage % Forces the first chapter to start on an odd page so it's on the right

\pagestyle{fancy} % Print headers again
 % Title
%%%%%%%%%%%%%%%%%%%%%%%%%%%%%%%%%%%%%%%%%%%%%%%%%%%%%%%%%%%
%%%% Main Document %%%%%%%%%%%%%%%%%%%%%%%%%%%%%%%%%%%%%%%%
%%%%%%%%%%%%%%%%%%%%%%%%%%%%%%%%%%%%%%%%%%%%%%%%%%%%%%%%%%%
% \section{Methodology}

\subsection{Study area}

The East Sea (Sea of Japan) was chosen in order to calculate chlorophyll-a concentration using the ocean color data. The study area is shown in Figure \ref{fig:Map_of_Research_Area}. as 34.00°N - 44.00°N, 127.00°E - 135.00°E which covers the whole East Sea (Sea of Japan) near Korea. The reasons for excluding the South Sea and the Yellow Sea from the study area are that it is difficult to calculate the chlorophyll-a concentration using the ocean color algorithm because of shallow depth. 

The depth of the East Sea is rapidly increase  from the east coast. And the East has a surface area of about 978,000 $\rm km^2$, a mean depth of 1,752 $\rm m $ and a maximum depth of 3,742 $\rm m$. The Terrains under the East Sea near Korean Peninsula is complex and the area of the continental shelf is steep.

\begin{figure}[h]

	\centering
{\includegraphics[width=0.95\textwidth]	{Map_of_Research_Area} }

\caption{Map showing study area.}
	\label{fig:Map_of_Research_Area}

\end{figure}

\newpage


\subsection{SeaWiFS data and Study period}

This research used SeaWiFS ocean color data to derive chlorophyll-a concentration since SeaWiFS is a satellite created to collect global ocean biological data. However, it is not able to obtain recent measurements  since SeaWiFS had only activated from September 1997 to December 2010. {Table \ref{table01}.} shows the mission characteristics of SeaWiFS and Table \ref{table02}. shows the characteristics of the SeaWiFS ocean color sensor \cite{hooker1992An}.

 \begin{table}[h!]
 	\caption{The mission characteristics of SeaWiFS.}
 	\label{table01}
 	\centering
 	\begin{tabular}{c  c}
  	\toprule%[width=0.9\textwidth]
  	 	Orbit Type	& Sun Synchronous at 705 km \\ \hline
  	%\midrule
 	Equator Crossing &	Noon +20 min, descending \\ \hline
 	Orbital Period &	99 minutes  \\ \hline
 	Swath Width &	2,801 km LAC/HRPT (58.3 degrees)  \\ \hline
 	Swath Width &	1,502 km GAC (45 degrees)  \\ \hline
 	Spatial Resolution &	1.1 km LAC, 4.5 km GAC  \\ \hline
 	Real-Time Data Rate &	665 kbps  \\ \hline
 	Revisit Time &	1 day  \\ \hline
 	Digitization &	10 bits  \\ 
 	\bottomrule
 	\end{tabular}
 \end{table}

 \begin{table}[h!]%[width=1.0\linewidth]
	\caption{Characteristics of the SeaWiFS ocean color sensor.}
	\label{table02}
	\centering
	\begin{tabular}{c  c  c  c  c}
		\toprule
		%\hline \setlength{\arrayrulewidth}{0.8pt}. 
		Band	& Wavelength FWHM[nm] & Saturation Radiance & Input Radiance & SNR\\ %\hline{5.0pt}
		\midrule
		1 & 402-422 & 13.63 & 9.10 & 499 \\ %\hline
		2 & 433-453 & 13.25 & 8.41 & 674  \\ %\hline
		3 & 480-500 & 10.50 & 6.56 & 667 \\ %\hline
		4 & 500-520 & 9.08 & 5.64 & 640  \\ %\hline
		5 & 545-565 & 7.44 & 4.57 & 596 \\ %\hline
		6 & 660-680 & 4.20 & 2.46 & 442  \\ %\hline
		7 & 745-785 & 3.00 & 1.61 & 455 \\ %\hline
		8 & 845-885 & 2.13 & 1.09 & 467  \\ %\hline
		 	\bottomrule
	\end{tabular}
\end{table}
 
 

The Ocean Biology Processing Group collects and processes data from Earth-viewing satellites. These data are organized in various ways reflecting different spatial, temporal, and parameter groupings. Over the years, certain terminology about remote sensing data has arisen to describe these organizational conventions. Level-0 is unprocessed data from remote sensing instruments at a full resolution. Level-1 data is obtained through radiometric and geometric calibration. Level-2 data is made of geophysical variables processed from level-1 data using developed algorithms \cite{feldman2017ocean}. In this research, chlorophyll-a concentration is the geophyscial variable we choose to process.

The SeaWiFS Level-2 Ocean Color chlorophyll-a Data Version 2014 can be downloaded from the NASA Goddard Space Flight Center (GSFC) Distributed Active Archive Center (DAAC). Processing Level-1 data to Level-2 data was conducted by NASA, using NASA SeaDAS program\cite{NASASeaFiWSdata}. NASA SeaDAS uses two algorithms to create chlorophyll-a concentration data from the Level-1 data of remote sensing reflectance ($\rm R_{rs}$); these algorithms are the OCx band ratio algorithm and the color index (CI) algorithm of Hu. 

The OCx band ratio algorithm use the ratio of two bands in a fourth-order polynomial equation in a relation with chlorophyll-a. It can be expressed as \eqref{eq:001}.
 
 \begin{equation}
 {\rm log_{10} (chlor_a) = a_0 + \sum_{i=1}^4 a_i ~ log_{10} \left( \frac{(R_{rs}(\lambda_{blue})}{(R_{rs}(\lambda_{green})} \right) ^i }
 \label{eq:001}
 \end{equation}
 
The coefficients $\rm a_0$ - $\rm a_4$ are values for each sensor acquired experimentally. The OCx algorithm was proved to be accurate by O’Reilly et. al\cite{o2000ocean}. The CI algorithm uses three bands and can be expressed as \eqref{eq:002}.

 \begin{equation}
{\rm CI = R_{rs} ({\lambda}_{green}) - [R_{rs} ({\lambda}_{blue}) + \frac {({\lambda}_{green} - {\lambda}_{blue})} {({\lambda}_{red} - {\lambda}_{blue})} * (R_{rs}{\lambda}_{red} - R_{rs} ({\lambda}_{blue}) ]}
\label{eq:002}
\end{equation}

The $\rm {\lambda}_{blue}$, $\rm {\lambda}_{green}$, $\rm {\lambda}_{red}$ are wavelengths closest to 443 $\rm nm$, 555 $\rm nm$ and 670 $\rm nm$, different for each sensor. According to Hu, C., Lee, Z., and Franz, B., for the lower concentration of chlorophyll-a (less than 0.25 $\rm mg/m^3$), it is more accurate to use CI algorithm \cite{hu2012chlorophyll}. The NASA SeaDAS software uses CI algorithm for chlorophyll retrievals below 0.15 $\rm mg/m^3$, and OCx band ratio algorithm for retrievals above 0.2 $\rm mg/m^3$. For the concentration between 0.15 $\rm mg/m^3$ and 0.2 $\rm mg/m^3$, it blends both algorithm.


The level-2 GAC data of the East sea (Sea of Japan) can be obtained from 1997 to 2010 but the level-2 LAC data can be obtained only from 2003 to 2006. Only a few LAC data files of the East Sea (Sea of Japan) are available after 2007. That is why the study period was set from 2003 to 2006. The information of on used data is shown in Table \ref{data_information}.

The level-3 LAC and GAC data are derived geophysical variables that have been aggregated onto a well-defined over a well-defined time period and projected onto a well-defined spatial grid.

 \begin{table}[h]
	\centering
	\caption{The number of Level-2 files for analysis.}
	\label{data_information}
	\begin{tabular}{c  c  c  }
		%\hline \setlength{\arrayrulewidth}{3.5pt}. 
		\toprule
		year & LAC data  & GAC data \\ %\hline
		\midrule
		2003 (N) & 482 & 490 \\ %\hline
		2004 (N) & 441 & 497 \\ %\hline
		2005 (N) & 270 & 494 \\ %\hline
		2006 (N) & 313 & 498 \\ %\hline
			\midrule
		total (N) & 1,506 & 1,979 \\ %\hline
		\bottomrule
	\end{tabular}
\end{table}




\hfill \break


 \subsection{Deriving Chlorophyll-a Concentration using LAC data and GAC data}
 
NASA SeaDAS is used to process the level-2 data of chlorophyll-a to level-3 data of monthly-mean/ 8-day-mean of chlorophyll-a concentration data. Monthly-mean data is created to show the overall tendency, while 8-day-mean data is created to see more specific tendency. Simple average method had been used to create the mean files. This method sums pixel values of chlorophyll-a at the same location and divides it by the number of data that have been compiled. This process also gives latitude longitude value to the pixels, creating a Standard Mapped Image (SMI). Chae, H. J., \& Park, K. (2009) used weighted average method to process data. However, according to Park, K. A., Park, J. E., Lee, M. S., \& Kang, C. K. (2012), both the weighted average method and the simple average method are suitable for processing SeaWiFS data in the East Sea (Sea of Japan). In addition, the more general method, the simple average method is used in this research

Since running each process using SeaDAS is inconvenient, so the process is automated by using python batch processing. This commands SeaDAS to repeat the process. Python can also create png files from the SMI. We use the obtained monthly-mean and 8-day-mean data to find the annual variability of chlorophyll-a concentration.

The effect of spatial resolution on the data was studied by comparing the LAC and the GAC data. The monthly average data was compared and the differences of the values were calculated. Further analysis was done on pixels with a value over 10 $\rm mg/m^3$. The maximum value of each month was also compared. These research were done to find out which data was more reliable. 와 같이 작성
%%%% 주의
%%%% 파일이 나뉠 때마다 자동으로 페이지넘김(\clearpage)가 됩니다. 
%%%% 따라서 subsection을 나누는 용도로는 사용하지 마십시오.
%%%% \include{sub/experiment} 와 같이...

%----------------------------------------------------------------------------------------
%	PART
%----------------------------------------------------------------------------------------

\part{대기}

%----------------------------------------------------------------------------------------
%	CHAPTER 1
%----------------------------------------------------------------------------------------

\chapterimage{chapter_head_2.pdf} % Chapter heading image

\chapter{대기과학 기초}

\section{대기과학에 사용되는 단위들}\index{Paragraphs of Text}

\lipsum[1-7] % Dummy text

%------------------------------------------------

\section{Citation}\index{Citation}

This statement requires citation \cite{book_key}; this one is more specific \cite[122]{article_key}.

%------------------------------------------------

\section{Lists}\index{Lists}

Lists are useful to present information in a concise and/or ordered way\footnote{Footnote example...}.

\subsection{Numbered List}\index{Lists!Numbered List}

\begin{enumerate}
	\item The first item
	\item The second item
	\item The third item
\end{enumerate}

\subsection{Bullet Points}\index{Lists!Bullet Points}

\begin{itemize}
	\item The first item
	\item The second item
	\item The third item
\end{itemize}

\subsection{Descriptions and Definitions}\index{Lists!Descriptions and Definitions}

\begin{description}
	\item[Name] Description
	\item[Word] Definition
	\item[Comment] Elaboration
\end{description}

%----------------------------------------------------------------------------------------
%	CHAPTER 2
%----------------------------------------------------------------------------------------

\chapter{In-text Elements}

\section{Theorems}\index{Theorems}

This is an example of theorems.

\subsection{Several equations}\index{Theorems!Several Equations}
This is a theorem consisting of several equations.

\begin{theorem}[Name of the theorem]
	In $E=\mathbb{R}^n$ all norms are equivalent. It has the properties:
	\begin{align}
		& \big| ||\mathbf{x}|| - ||\mathbf{y}|| \big|\leq || \mathbf{x}- \mathbf{y}||\\
		&  ||\sum_{i=1}^n\mathbf{x}_i||\leq \sum_{i=1}^n||\mathbf{x}_i||\quad\text{where $n$ is a finite integer}
	\end{align}
\end{theorem}

\subsection{Single Line}\index{Theorems!Single Line}
This is a theorem consisting of just one line.

\begin{theorem}
	A set $\mathcal{D}(G)$ in dense in $L^2(G)$, $|\cdot|_0$. 
\end{theorem}

%------------------------------------------------

\section{Definitions}\index{Definitions}

This is an example of a definition. A definition could be mathematical or it could define a concept.

\begin{definition}[Definition name]
	Given a vector space $E$, a norm on $E$ is an application, denoted $||\cdot||$, $E$ in $\mathbb{R}^+=[0,+\infty[$ such that:
	\begin{align}
		& ||\mathbf{x}||=0\ \Rightarrow\ \mathbf{x}=\mathbf{0}\\
		& ||\lambda \mathbf{x}||=|\lambda|\cdot ||\mathbf{x}||\\
		& ||\mathbf{x}+\mathbf{y}||\leq ||\mathbf{x}||+||\mathbf{y}||
	\end{align}
\end{definition}

%------------------------------------------------

\section{Notations}\index{Notations}

\begin{notation}
	Given an open subset $G$ of $\mathbb{R}^n$, the set of functions $\varphi$ are:
	\begin{enumerate}
		\item Bounded support $G$;
		\item Infinitely differentiable;
	\end{enumerate}
	a vector space is denoted by $\mathcal{D}(G)$. 
\end{notation}

%------------------------------------------------

\section{Remarks}\index{Remarks}

This is an example of a remark.

\begin{remark}
	The concepts presented here are now in conventional employment in mathematics. Vector spaces are taken over the field $\mathbb{K}=\mathbb{R}$, however, established properties are easily extended to $\mathbb{K}=\mathbb{C}$.
\end{remark}

%------------------------------------------------

\section{Corollaries}\index{Corollaries}

This is an example of a corollary.

\begin{corollary}[Corollary name]
	The concepts presented here are now in conventional employment in mathematics. Vector spaces are taken over the field $\mathbb{K}=\mathbb{R}$, however, established properties are easily extended to $\mathbb{K}=\mathbb{C}$.
\end{corollary}

%------------------------------------------------

\section{Propositions}\index{Propositions}

This is an example of propositions.

\subsection{Several equations}\index{Propositions!Several Equations}

\begin{proposition}[Proposition name]
	It has the properties:
	\begin{align}
		& \big| ||\mathbf{x}|| - ||\mathbf{y}|| \big|\leq || \mathbf{x}- \mathbf{y}||\\
		&  ||\sum_{i=1}^n\mathbf{x}_i||\leq \sum_{i=1}^n||\mathbf{x}_i||\quad\text{where $n$ is a finite integer}
	\end{align}
\end{proposition}

\subsection{Single Line}\index{Propositions!Single Line}

\begin{proposition} 
	Let $f,g\in L^2(G)$; if $\forall \varphi\in\mathcal{D}(G)$, $(f,\varphi)_0=(g,\varphi)_0$ then $f = g$. 
\end{proposition}

%------------------------------------------------

\section{Examples}\index{Examples}

This is an example of examples.

\subsection{Equation and Text}\index{Examples!Equation and Text}

\begin{example}
	Let $G=\{x\in\mathbb{R}^2:|x|<3\}$ and denoted by: $x^0=(1,1)$; consider the function:
	\begin{equation}
		f(x)=\left\{\begin{aligned} & \mathrm{e}^{|x|} & & \text{si $|x-x^0|\leq 1/2$}\\
			& 0 & & \text{si $|x-x^0|> 1/2$}\end{aligned}\right.
	\end{equation}
	The function $f$ has bounded support, we can take $A=\{x\in\mathbb{R}^2:|x-x^0|\leq 1/2+\epsilon\}$ for all $\epsilon\in\intoo{0}{5/2-\sqrt{2}}$.
\end{example}

\subsection{Paragraph of Text}\index{Examples!Paragraph of Text}

\begin{example}[Example name]
	\lipsum[2]
\end{example}

%------------------------------------------------

\section{Exercises}\index{Exercises}

This is an example of an exercise.

\begin{exercise}
	This is a good place to ask a question to test learning progress or further cement ideas into students' minds.
\end{exercise}

%------------------------------------------------

\section{Problems}\index{Problems}

\begin{problem}
	What is the average airspeed velocity of an unladen swallow?
\end{problem}

%------------------------------------------------

\section{Vocabulary}\index{Vocabulary}

Define a word to improve a students' vocabulary.

\begin{vocabulary}[Word]
	Definition of word.
\end{vocabulary}

%----------------------------------------------------------------------------------------
%	PART
%----------------------------------------------------------------------------------------

\part{Part Two}

%----------------------------------------------------------------------------------------
%	CHAPTER 
%----------------------------------------------------------------------------------------

\chapterimage{chapter_head_1.pdf} % Chapter heading image

\chapter{일기예보 과정}

\section{예보가 나오는 과정}

예보가 나오기까지는 기상실황파악 → 자료수집 → 분석 → 예보작성 → 통보 과정을 거친다. 

%------------------------------------------------

\section{기상실황 파악}

\subsection{지상기상 관측}
전국 76개소의 기상관서에서 하늘상태, 시정 등의 목측 (目測 ) 요소를 관측하고 있으며, 기온, 습도, 강수량, 바람, 기압 등은 자동기상관측장비를 이용하여 1분 간격으로 관측되고 있다. 또한, 기상관서가 없는 500여 소에서는 방재용 자동기상관측장비를 이용하여 기온, 풍향, 풍속, 강수량, 강수 유무를 1분 간격으로 관측하여 기상실황을 감시하고 있다.

\subsection{항공기상관측}
전국 공항기상관서에서는 바람, 시정, 운고, 기온, 기압 등의 항공기상관측요소를 매30분 또는 1시간 간격으로 관측하며, 활주로에 설치된 공항기상관측장비에 의해 기상요소들이 매분 자동 관측된다. 특히 인천, 제주, 양양, 울산 등의 공항에서는 이·착륙 항공기에 영향을 미치는 저층난류를 탐측하기 위하여 저층난류경보장치를 운영하고 있다.

\subsection{고층기상관측}
고층기상관측은 지상보다 높은 상층 대기의 상태를 관측하는 것으로, 기상청은 레윈존데 관측, 수직측풍장비 관측을 수행한다. 레윈존데 관측은 기구에 라디오존데를 매달아 지상으로부터 약 35 km(5 hPa)까지의 고도별 기압, 기온, 습도, 풍향, 풍속을 00UTC와 12UTC에 관측하며, 수직측풍장비 관측은 UHF나 VHF 파장의 전파를 상층대기로 방사하고 바람과 함께 이동하는 난류에 산란되어 다시 수신되는 전파신호로 바람을 10분 간격으로 관측한다.

\subsection{해양기상관측}
해양기상관서, 해양기상관측부이, 해양기상영상감시시스템을 통하여 풍향·풍속, 기온, 수온, 기압, 파고 등을 관측하며, 먼바다의 기상현상 관측 및 부이 관리를 위하여 기상관측선을 운영하고 있다.

\subsection{기상위성관측}
기상위성은 우주공간에서 지구의 기상변화를 관측한다. 기상청은 정지궤도기상위성과 극궤도기상위성의 자료를 직접 수신하여 처리하여 예보를 위해 사용하고, 국민에게도 공개하고 있다. 이를 위해 기상위성 수신처리분석시스템을 서울, 문산, 서산에 설치하여 운영중이다.

\subsection{기상레이더 관측}
도플러 기상레이더를 설치하여 한반도에서 발생하는 악기상을 관측하여 예보에 활용하고 있다. 또한 일본 기상청과 공군의 레이더 자료도 수신하여 기존영상과 합성하여 종합적으로 활용하고 있다.


\section{자료수집}
통신용컴퓨터를 이용하여 국내기상자료와 외국에서 송신되는 각종기상자료를 수집, 편집, 가공하여 분석용 컴퓨터로 보낸다. 국내·외에서 수집된 관측자료로부터 수치예보모델을 이용하여 예상일기도를 생산한다. 이러한 수치예보모델의 운용을 위해 슈퍼컴퓨터가 사용된다.


%----------------------------------------------------------------------------------------
%	CHAPTER 
%----------------------------------------------------------------------------------------

\chapterimage{chapter_head_1.pdf} % Chapter heading image

\chapter{기상청(www.kma.go.kr) 일기 예보}

\section{기상청 일기 예보의 종류}


\subsection{초단기 동네 예보}
초단기예보는 현재부터 앞으로 3시간까지, 실황(날씨, 기온, 습도 등 7개 요소)과 예보(강수형태, 하늘상태, 강수량 등 3개 요소)를 1시간 간격으로 동네예보를 기반으로 매 시각 30분에 발표한다. 초단기예보는 짧은 시간에 발생·소멸하는 기상현상에 대해 신속하게 대응하여 재해예방에 최선을 다하고자 2010년 6월 15일부터 홈페이지를 통해 제공하고 있다.

\subsection{동네 예보}
대상기간과 구역을 시ㆍ공간적으로 세분화하여 발표하는 예보로 기온, 최고기온, 최저기온, 강수형태, 강수확률, 12시간강수량, 12시간적설, 하늘상태, 습도, 풍향, 풍속, 파고 등을 예보한다. 동네예보는 3시간 간격으로 1일에 8회 예보하며 예보구간도 역시 3시간 단위로 48시간까지 예보한다. 

\subsection{주간 예보}
기상전망, 예보구역별 육상 및 해상 날씨, 지점별 기온, 파고에 대한 48시간 이후부터  6일간의 예보로 일 2회 발표(06시, 18시)하는 주간예보(모레부터 6일간)가 계속 유지될 가능성에 대한 신뢰도 정보를 3단계로 구분하여 제공(육상)한다.

\begin{table}[h]
	\center
\begin{tabular}{c|c}
	\hline 
신뢰도  &	내용	  \\ 	\hline 
높음  & 다음날 발표 주간예보가 계속 유지될 가능성이 높음  \\  \hline 
보통 & 다음날 발표 주간예보가 계속 유지될 가능성이 있음  \\ 	\hline 
낮음 & 다음날 발표 주간예보가 계속 유지될 가능성이 낮음   \\ 	\hline 
\end{tabular} 
	\caption{신뢰도와 의미}
\end{table}

\subsection{주말 예보}
토요일과 일요일의 기상 개황, 일별 날씨, 야외활동 지수 등의 정보를 제공한다. 화요일 19시부터 금요일 24시까지 제공되며, 매일 19시에 발표한다.

\subsection{장기 예보}
장기예보는 11일 이상에 대한 예보를 일컬으며 순별·월별 기압계 동향 및 전망, 기온·강수량 예보 등을 발표한다. 예보구역은 한반도 12개 권역(서울·인천·경기도, 강원도 영서, 강원도 영동, 대전·충청남도, 충청북도, 광주·전라남도, 전라북도, 부산·울산·경상남도, 대구·경상북도, 제주도, 평안남북도·황해도, 함경남북도)이며, 월 3회 발표되는 1월 전망과 월 1회 발표되는 3개월 전망이 있다. 그 외 연 4회 발표되는 기후전망은 다음다음 계절에 대한 전망으로 봄철 기후전망은 11월 23일 경에, 여름철 기후전망은 2월 23일 경에, 가을철 기후전망은 5월 23일 경에 겨울철 기후전망은 8월 23일 경에 발표한다.



\part{대기과학 실험}


%----------------------------------------------------------------------------------------
%	CHAPTER
%----------------------------------------------------------------------------------------

\chapter{기초 실험}

\section{빗방울의 낙하 속도}\index{빗방울의 낙하 속도}

Rain is the liquid form of precipitation on Earth. It is part of the hydrologic cycle that begins when water evaporates and forms clouds in the atmosphere. The water that forms these clouds is frozen and vaporized. Once enough water has evaporated, it is then released in the form of droplets of rain back to the surface of the Earth.

The average size of a raindrop is 6 millimeters in diameter, about the size of a housefly. Of course all raindrops vary in size due to the strength of a specific rainstorm, but this is considered a reasonable value of a typical raindrop. When a raindrop falls to the surface of the Earth, it is acted on by two main forces, gravity and drag. A stationary raindrop initially experiences an acceleration due to gravity of 9.8 m/s2, as would any falling body. As gravity increases the speed of the raindrop in its descent, drag retards the downward acceleration of the raindrop. Usually, air resistance that comes in contact with the water molecules as they fall causes the drag. The combination of these two forces causes a raindrop to reach a terminal velocity when the drag force is approximately equal to the weight of the raindrop. At this point, a raindrop experiences no further acceleration and therefore falls at a constant velocity.

The magnitude of the terminal velocity of an object is also affected by its orientation. A common misconception is the shape of the raindrop. It is often depicted as pointy and lopsided. However, research has found the shape of a raindrop to be rather spherical or slightly flattened on the bottom by airflow like a hamburger bun.

The terminal velocity of a 6-millimeter raindrop was found to be approximately 10 m/s. This value has been found to vary between 9 m/s and 13 m/s when measurements were taken on different days. The variance has been contributed to different air temperatures and pressures. In comparison, a human being falling to the surface of the Earth experiences a drastically larger terminal velocity of approximately 56 m/s.


How Fast Is Falling Rain? \cite{evan2007}

Let the physics begin. You might think: hey, wont’ the speed depend on how high the water started? Well, it would if air resistance on the water drop were not important. However, I suspect that the rain will fall at terminal velocity. Terminal velocity is the case when the air resistance on the object is equal to the gravitational force on the object. When this happens, the net force is zero (the zero vector) and the object falls at a constant speed.

Here is a diagram of a water drop at terminal speed.

Untitled 1
Since the air resistance force depends on the speed of the object (but the gravitational force does not), there is one speed for which these two forces add up to the zero vector. Near the surface of the Earth, the magnitude of the gravitational force can be modeled as:

La te xi t 1 4

Where g is the local gravitational field (not the acceleration due to gravity – that is a non-good name for it). And what about the air resistance? It can probably be modeled as:

La te xi t 1 5

Where:

ρ is the density of air (about 1.2 kg/m3).
A is the cross-sectional area of the object. If the object was a sphere, this area would be the area of a circle.
C is the drag coefficient. This depends on the shape of the object. A cone and a flat circle will have the same A, but different drag coefficients.
v is the magnitude of the velocity of the object with respect to the air.
It won’t matter for this case too much, but the direction of the air resistance force is in the opposite direction to the velocity.
At terminal velocity, the magnitudes of these two forces will be equal. I can write that as:

La te xi t 1 6

Now, what about the mass (m)? Let me assume that it is made of water (like most rain) and is spherical (even though that is not likely – it would probably be “rain drop shaped”). If I call the density of water ρw and the radius of the drop r, then the mass would be:

La te xi t 1 7

Putting this into the “weight = air resistance” expression above as well as an expression for the cross-sectional area in terms of r, I get:

La te xi t 1 8

The cool thing here is that the terminal speed of the water drop depends on the size (radius). Larger drops will have a larger terminal velocity. So, could you just make a water melon sized water drop? No. Why not? Because at some point, the force from the air on the drop is going to break the water drop apart. The surface tension holding the drop together just won’t be enough to maintain its drop status.

Then how big can it get? I have no idea. Oh, and then there is the problem of real drop instead of spherical drops. Let me look at that first. Wikipedia lists the coefficient of drag for a smooth sphere as 0.1. A rain drop should be less than this – but how much less? Well, a rain drop would take some of the water to form some sort of tail. This would decrease the cross sectional area as well as decrease the drag coefficient. I am not sure how to calculate the volume of a non-spherical rain drop, so for now I will just use a spherical drop with a drag coefficient of 0.08. I know that is wrong, but it will give me an idea about the terminal speed.

Now, how big should it be? How about I don’t decide. Instead I will plot the terminal speed for a range of rain drop sizes. Let me look at drops from 0.5 mm to 5 mm. Here is that plot.

Raindrop.png

Well, the original question asked about the speeds in units of miles per hour. Here is the same plot but with different units.

Raindrop 2.png

Based on my estimations, 17 mph would be on the low end – but possible. It could be likely that I grossly overestimated the size of a raindrop.

Homework: Yes, there is homework. If the rain drop has a radius of 0.5 mm, from how high would it have to drop to get pretty close to the terminal velocity?


UPDATE


As usual, I rush into things without exploring things in more depth. My assumption of a raindrop shaped raindrop appears to be bogus. Who would have guessed that? Anyway, here are some very useful links from commenters (Jens and Charles) and a large thanks to them.

A German kid’s video showing the shape of a raindrop (I think).
A nice summary of findings for falling rain drops.
Terminal Velocity of Rain Drops Aloft – paper from the Journal of Applied Meteorology (pdf)
Here is another link from @swansontea: Bad Rain: Raindrops are not tear drop shaped.

%----------------------------------------------------------------------------------------
%	CHAPTER 
%----------------------------------------------------------------------------------------

\chapter{대기의 안정도와 단열선도}

\section{단열선도}
1930년대부터 고층기상관측이 실시되면서 대기를 입체적으로 파악하게 되었다. 고층기상관
측 자료를 신속하게 정리·분석하여 한 관측 지점의 연직방향 열역학적 특성을 파악하는데 단
열선도가 이용되고 있다. 단열선도는 복잡한 수식으로 계산해야 하는 기상 현상이나 기상요소
를 간단히 구할 수 있도록 고안된 도표이다. 단열선도에는 여러 종류가 있으나 어느 것이나 건
조 단열선, 습윤 단열선, 포화 혼합비선, 등온선, 등압선의 5선이 그려져 있다. 온도와 기압이
기울어져 Skew T-log P diagram라는 명명된 단열선도<그림 Ⅲ-18>이 가장 많이 쓰이고 있다.

\begin{figure}[h]
	\centering
	\includegraphics[width=0.8\linewidth]{skewT-logP-1}
	\caption{단열선도의 구성 선들}
	\label{fig:skewt-logp}
\end{figure}

단열선도에서 구할 수 있는 기상요소들은 다음과 같다.
① 혼합비 : 건조공기 1 kg 속에 혼합된 수증기의 질량(g)으로 Skew T-log P diagram에서
는 알고자 하는 고도의 이슬점 온도를 지나는 포화혼합비선의 값을 읽으면 된다.
② 상대습도 : R.H(%) = 실제 혼합비/ 포화 혼합비  ×100이므로 Skew T-log P diagram에서는
R.H(%)= 이슬점 온도에 해당하는 혼합비/ 기온에 해당하는 혼합비 
이 된다.
③ 상승응결고도(LCL, Lifting Condensation Level) : 온도점에서 건조 단열선을 따라 올라간
선과 이슬점에서 포화 혼합비선을 따라 올라간 선이 만나는 점의 고도이다.
④ 대류응결고도(CCL, Convection Condensation Level) : 상승응결고도에서 포화 혼합비선을
따라 올라간 선과 기온 곡선이 만나는 점의 고도이다


\section{사용 실례}\index{단열선도 사용 실례}

예를 들어 기압이 1000 hPa, 기온 3 ℃, 상대습도 70%인 공기덩어리(A점)가 위로 상승한다
고 가정하자. 기온 3 ℃에서의 포화혼합비를 그림에서 구하면 5 g이므로 이 공기의 혼합비는
그 70%인 3.5 g이다. 이 값은 공기 중의 수증기가 응결되어 줄어 들거나 또는 외부로부터 공
급되지 않는 한 변하지 않는다. 이 공기덩어리가 상층으로 올라가면 단열적으로 변화하므로
단열선도의 건조 단열선을 따라 기압이 내려간다. 그리고 925 hPa 부근에서 3.5 g의 혼합비선
과 만난다. 이것은 이 공기덩어리의 상대습도가 점차 증가하여 여기서 100% 즉, 포화에 도달
된다는 것을 의미한다. 따라서 그 이상의 고도로 올라가면 공기덩어리 중의 수증기가 응결하
면서 변화하게 되므로 기온이 습윤 단열선을 따라 변해간다. 수증기가 응결하기 시작하는 이
고도를 응결고도 또는 구름이 생성되는 높이이므로 운저고도(雲底高度), 운고라 한다. 이 예에
서 응결고도는 925 hPa 고도이며 대략 지상 750 m 정도가 된다.

\begin{figure}[h]
	\centering
	\includegraphics[width=0.8\linewidth]{skewT-logP-2}
	\caption{단열선도 이용 실례}
	\label{fig:skewt-logp-1}
\end{figure}


\section{대기 안정도 분석}\index{단열선도를 이용한 대기 안정도 분석}

단열선도를 이용하면 대기층의 안정 상태를 쉽게 파악할 수 있다. 고층관측 결과를 단열선
도에 기입했을 때 대기의 온도가 <그림 Ⅲ-19>의 굵은 선 B와 같이 변하였다고 하자. 지상의
공기덩어리를 상층으로 가지고 올라갔을 때의 상황을 조사해 보면, 그림에서 파선과 같이 되
므로 이 공기덩어리의 기온은 주위 공기의 기온보다 높아져 밀도가 작아지므로 부력을 받아
계속 상승하려 한다. 그래서 이런 상태의 대기성층을 불안정이라 한다. 이런 때는 대류가 일
어난다. 여기서 알 수 있는 것처럼 대기의 실제 기온감률이 100 m당 1 ℃ 이상이면 그 기층은
불안정하다는 것을 알 수 있으며, 특히 이 경우를 절대 불안정이라 한다.
대기의 온도 변화가 <그림 Ⅲ-19>에서 굵은 선 C와 같았다고 할 경우에는, 그 기층의 공기
가 포화되지 않고 건조하다면 건조 단열선을 따라 변하므로 상공으로 올라가면서 밀도가 본래
의 대기 밀도보다 커지므로 안정하게 된다. 그런데 공기가 포화되어 있으면 습윤 단열선을 따
라 변하므로 상공에 올라가 주위 공기보다 기온이 높아져 불안정해진다. 이상으로 보면 수증
기 함유량에 따라 안정하기도 하고 불안정하기도 한다. 이것을 조건부 불안정이라 한다.
<그림 Ⅲ-19>에서 D와 같은 경우에는 공기가 건조한 경우나 습윤한 경우에 관계 없이 위로
올라간 공기덩어리는 주위 공기보다 기온이 낮아 안정적이다. 이것을 기온감률로 표현하면 실
제 기온감률이 습윤단열변화 할 때보다 작은 경우로서 대기의 기온감률이 100 m당 0.5 ℃ 미
만인 경우이다. 이를 절대 안정이라 한다.
만일 실제 기온의 고도변화가 100 m 당 1 ℃라면 공기덩어리가 건조단열 변화하면서 상승
해도 주위와의 기온차가 생기지 않는다. 또 기온의 고도변화가 습윤 단열선을 따르는 상태에
서는 포화된 공기덩어리는 상승해도 주위와의 기온차가 없다. 즉, 안정도 불안정도 아닌 중립
상태가 된다.
<그림 Ⅲ-19>의 E의 경우는 파선을 따라 변화한다. 따라서 이 경우 885hPa이하에서는 안정
이나 그 이상 끌어올리면 주위보다 기온이 높아져 불안정이 된다. 즉 요란이 작으면 안정 상태
를 유지하지만, 요란이 크면 불안정하게 된다. 이러한 상태를 잠재 불안정이라 한다.


%------------------------------------------------



\begin{definition}[단열선도를 이용한 대기 분석]
	
	
	열선도 중에서 가장 널리 이용되고 있는 Skew T-log P diagram 상에 실제 관측 자료를 표시하여 여러 가지 기상현상을 구해보고, 이를 통해 대기를 입체적으로 분석해 보자.
	
	1. 다음 표는 고층 기상 관측 자료이다.
	
	
	\begin{tabular}{c|c|c|c|c|c}
		\hline 
		기압(hPa) & 기온(℃) & 이슬점  온도 & 혼합비(g/kg) & 포화혼합비 & 상대습도(\%) \\ \hline 
		1000 & 25 & 14 &	 &  &  \\ 
		\hline 
		900 & 18 & 6 &  &  &  \\ 
		\hline 
		800 & 8 & -1 &  &  &  \\ 
		\hline 
		700 & 3 & -12  &  &  &  \\ 
		\hline 
		600 & 0 & -10 &  &  &  \\ 
		\hline 
		500 & -13 & -151 &  &  &  \\ 
		\hline 
		400 & -16 &-40	&  &  &  \\ 
		\hline 
	\end{tabular} 
	
	
\end{align}
\end{definition}

%------------------------------------------------


%----------------------------------------------------------------------------------------
%	PART
%----------------------------------------------------------------------------------------

\part{Part Two}

%----------------------------------------------------------------------------------------
%	PART
%----------------------------------------------------------------------------------------
\usechapterimagetrue
\part{Part One}

%----------------------------------------------------------------------------------------
%	CHAPTER 1
%----------------------------------------------------------------------------------------

\chapterimage{chapter_head_2.pdf} % Chapter heading image

\chapter{Text Chapter}

\section{Paragraphs of Text}\index{Paragraphs of Text}

\lipsum[1-7] % Dummy text

%------------------------------------------------

\section{Citation}\index{Citation}

This statement requires citation \cite{book_key}; this one is more specific \cite[122]{article_key}.

%------------------------------------------------

\section{Lists}\index{Lists}

Lists are useful to present information in a concise and/or ordered way\footnote{Footnote example...}.

\subsection{Numbered List}\index{Lists!Numbered List}

\begin{enumerate}
\item The first item
\item The second item
\item The third item
\end{enumerate}

\subsection{Bullet Points}\index{Lists!Bullet Points}

\begin{itemize}
\item The first item
\item The second item
\item The third item
\end{itemize}

\subsection{Descriptions and Definitions}\index{Lists!Descriptions and Definitions}

\begin{description}
\item[Name] Description
\item[Word] Definition
\item[Comment] Elaboration
\end{description}

%----------------------------------------------------------------------------------------
%	CHAPTER 2
%----------------------------------------------------------------------------------------

\chapter{In-text Elements}

\section{Theorems}\index{Theorems}

This is an example of theorems.

\subsection{Several equations}\index{Theorems!Several Equations}
This is a theorem consisting of several equations.

\begin{theorem}[Name of the theorem]
In $E=\mathbb{R}^n$ all norms are equivalent. It has the properties:
\begin{align}
& \big| ||\mathbf{x}|| - ||\mathbf{y}|| \big|\leq || \mathbf{x}- \mathbf{y}||\\
&  ||\sum_{i=1}^n\mathbf{x}_i||\leq \sum_{i=1}^n||\mathbf{x}_i||\quad\text{where $n$ is a finite integer}
\end{align}
\end{theorem}

\subsection{Single Line}\index{Theorems!Single Line}
This is a theorem consisting of just one line.

\begin{theorem}
A set $\mathcal{D}(G)$ in dense in $L^2(G)$, $|\cdot|_0$. 
\end{theorem}

%------------------------------------------------

\section{Definitions}\index{Definitions}

This is an example of a definition. A definition could be mathematical or it could define a concept.

\begin{definition}[Definition name]
Given a vector space $E$, a norm on $E$ is an application, denoted $||\cdot||$, $E$ in $\mathbb{R}^+=[0,+\infty[$ such that:
\begin{align}
& ||\mathbf{x}||=0\ \Rightarrow\ \mathbf{x}=\mathbf{0}\\
& ||\lambda \mathbf{x}||=|\lambda|\cdot ||\mathbf{x}||\\
& ||\mathbf{x}+\mathbf{y}||\leq ||\mathbf{x}||+||\mathbf{y}||
\end{align}
\end{definition}

%------------------------------------------------

\section{Notations}\index{Notations}

\begin{notation}
Given an open subset $G$ of $\mathbb{R}^n$, the set of functions $\varphi$ are:
\begin{enumerate}
\item Bounded support $G$;
\item Infinitely differentiable;
\end{enumerate}
a vector space is denoted by $\mathcal{D}(G)$. 
\end{notation}

%------------------------------------------------

\section{Remarks}\index{Remarks}

This is an example of a remark.

\begin{remark}
The concepts presented here are now in conventional employment in mathematics. Vector spaces are taken over the field $\mathbb{K}=\mathbb{R}$, however, established properties are easily extended to $\mathbb{K}=\mathbb{C}$.
\end{remark}

%------------------------------------------------

\section{Corollaries}\index{Corollaries}

This is an example of a corollary.

\begin{corollary}[Corollary name]
The concepts presented here are now in conventional employment in mathematics. Vector spaces are taken over the field $\mathbb{K}=\mathbb{R}$, however, established properties are easily extended to $\mathbb{K}=\mathbb{C}$.
\end{corollary}

%------------------------------------------------

\section{Propositions}\index{Propositions}

This is an example of propositions.

\subsection{Several equations}\index{Propositions!Several Equations}

\begin{proposition}[Proposition name]
It has the properties:
\begin{align}
& \big| ||\mathbf{x}|| - ||\mathbf{y}|| \big|\leq || \mathbf{x}- \mathbf{y}||\\
&  ||\sum_{i=1}^n\mathbf{x}_i||\leq \sum_{i=1}^n||\mathbf{x}_i||\quad\text{where $n$ is a finite integer}
\end{align}
\end{proposition}

\subsection{Single Line}\index{Propositions!Single Line}

\begin{proposition} 
Let $f,g\in L^2(G)$; if $\forall \varphi\in\mathcal{D}(G)$, $(f,\varphi)_0=(g,\varphi)_0$ then $f = g$. 
\end{proposition}

%------------------------------------------------

\section{Examples}\index{Examples}

This is an example of examples.

\subsection{Equation and Text}\index{Examples!Equation and Text}

\begin{example}
Let $G=\{x\in\mathbb{R}^2:|x|<3\}$ and denoted by: $x^0=(1,1)$; consider the function:
\begin{equation}
f(x)=\left\{\begin{aligned} & \mathrm{e}^{|x|} & & \text{si $|x-x^0|\leq 1/2$}\\
& 0 & & \text{si $|x-x^0|> 1/2$}\end{aligned}\right.
\end{equation}
The function $f$ has bounded support, we can take $A=\{x\in\mathbb{R}^2:|x-x^0|\leq 1/2+\epsilon\}$ for all $\epsilon\in\intoo{0}{5/2-\sqrt{2}}$.
\end{example}

\subsection{Paragraph of Text}\index{Examples!Paragraph of Text}

\begin{example}[Example name]
\lipsum[2]
\end{example}

%------------------------------------------------

\section{Exercises}\index{Exercises}

This is an example of an exercise.

\begin{exercise}
This is a good place to ask a question to test learning progress or further cement ideas into students' minds.
\end{exercise}

%------------------------------------------------

\section{Problems}\index{Problems}

\begin{problem}
What is the average airspeed velocity of an unladen swallow?
\end{problem}

%------------------------------------------------

\section{Vocabulary}\index{Vocabulary}

Define a word to improve a students' vocabulary.

\begin{vocabulary}[Word]
Definition of word.
\end{vocabulary}

%-----------------------------------------------------
%   감사의 글
%-----------------------------------------------------
\begin{acknowledgements}
\addcontentsline{toc}{section}{감사의 글}  %%% TOC에 표시
정말 감사합니다.
\end{acknowledgements}

%-----------------------------------------------------
%   연구활동 
%-----------------------------------------------------
\begin{researches}
\addcontentsline{toc}{section}{연구활동}  %%% TOC에 표시
\begin{itemize}
\item{2011학년도 교내 R\&E 발표대회에서 장려상 수상}
\item{2012학년도 교내 R\&E 발표대회에서 장려상 수상}
\item{2013학년도 교내 R\&E 발표대회에서 장려상 수상}
\item{2014학년도 교내 R\&E 발표대회에서 장려상 수상}
\item{2015학년도 교내 R\&E 발표대회에서 장려상 수상}
\item{2016학년도 교내 R\&E 발표대회에서 장려상 수상}
\item{2017학년도 교내 R\&E 발표대회에서 장려상 수상}
\item{2018학년도 교내 R\&E 발표대회에서 장려상 수상}
\item{2019년 노벨 물리학상 수상}
\end{itemize}
\end{researches}

\bibliography{bibfile} % 참고문헌
% BibTeX 코드 쉽게 얻어오는 방법 %
% Google Scholar 에서 검색한 결과에서 `인용'을 클릭한다.
% BibTeX 코드를 얻고자 한다면, 하단의 `BibTeX' 을 클릭.
% 코드가 나온다. Ctrl+A, Ctrl+C로 복사, bibfile에 붙여넣기.


\end{document}
