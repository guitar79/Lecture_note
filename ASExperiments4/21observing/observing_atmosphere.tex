
%----------------------------------------------------------------------------------------
%	PART
%----------------------------------------------------------------------------------------
\part{실험}

\chapterimage{chapter_head_1.pdf} % Chapter heading image
\chapter{대기 관측}


\section{대기 관측 개관}\index{대기 관측 개관}

\subsection{필요성}\index{필요성}

대기 과학이라는 학문이 다른 분야의 그것과 구분되는 요소는 과거와 현재의 대기 상태를 토대로 미래의 대기의 변화를 예측하는 것이라고 할 수 있다.따라서 정확한 일기 예보를 위해서는 대기의 열 역학적, 운동 역학적 특성을 명확하게 이해하고, 적절한 분석 기법, 모델 자료 등을 활용하여 미래의 대기 상태를 예측하는 능력이 필요하다. 그러기 위해서는 현재 대기 상태에 대한 정확한 정보를 가지고 있다는 중요한 전제가 따르게 된다. 즉, 과거와 현재의 대기 상태에 대한 정확한 정보가 없이는 어떠한 뛰어난 대기과학적 지식이나 분석 및 예보 도구도 무용지물이 될 수밖에 없다.

세계기상기구(WMO)는 보다 정확하고 표준화된 대기관측에 대한 일반적인 지침을 제공하기 위해 ⌜대기관측 측기와 방법에 관한 지침(Guide to Meteorological Instruments and Methods of Observation)⌟을 발행하여 현업 기관에서 활용하도록 하고 있다. 이 지침서는 WMO 산하에 있는 기상대에서 관측해야 하는 기상 요소들과 이에 필요한 측기들 그리고 이들을 운영 유지하는데 필요한 상세한 내용들을 포함하고 있다. 

\subsection{대표성}\index{대표성}

대기관측의 대표성이란 특정한 목적에 필요한 기상 변수들의 값이 얼마나 정확하게 설명되어 지는가를 의미한다. 따라서 대표성이란 어떤 특정한 관측 값을 의미하는 것이 아니라, 관측 도구, 특정한 적용에 필요한 관측 주기와 대푯값 여부 등을 종합 판정해서 결정되어진다. 예를 들어, 종관 관측은 일반적으로 관측소 주변 100km 지역까지를 대표할 수 있는 곳에서 이루어져야 하는 반면, 중규모나 국지규모 기상에 필요한 관측은 10km 이하의 대표성을 가져도 된다.

대표성은 평균을 위한 시간적 공간적 규모, 관측소 밀도, 기상 현상의 수평 규모 등에 따라서 결정되는데, 농업기상의 경우에는 아주 작고, 전구 규모의 장기 예측을 위해서는 상대적으로 큰 관측 주기와 수평 해상도를 가지게 된다. 예보 규모는 기상 현상의 시간 규모와 밀접하게 관련되어 있다.
초단기 예보의 경우에는 시간적 공간적 규모가 작은 기상 현상들을 탐지해야 하기 때문에 그 만큼 좁은 지역에 조밀한 관측망을 구축해서 짧은 주기로 관측해야 빠르게 발달하고 소멸하는 제대로 관측하고 예측할 수 있는 것이다. 

WMO 보고서와 다양한 연구 결과를 토대로 기상 현상의 수평 규모는 다음과 같이 구분할 수 있다.
\begin{itemize}
	\item 미세규모(Microscale) : 100m 미만의 수평 규모를 가지며 주로 농업기
상학에 적용되고 주요한 관측 요소는 증발량이 있다.
	\item(b) 국소(Toposcale) 혹은 국지규모(Local scale) : 100m에서 3km까지의
수평규모를 가지며, 대기 오염이나 토네이도 활동 등과 관련되어 있다.
	\item(c) 중규모(Mesoscale) : 3km에서 100km의 수평규모를 가지며 뇌우, 해
륙풍, 산악풍 등이 이에 해당한다.
	\item(d) 대규모(Large scale) : 100km에서 3,000km의 수평규모를 가지며 전
선, 다양한 저기압, 구름 무리 등이 이 규모에 해당한다.
	\item(e) 행성규모(Planetary scale) : 3,000km 이상의 수평규모를 가지며, 대
류권 상층의 장파가 이에 해당한다.

대표성을 확보할 수 있는 좋은 대기 관측은 관측 기술, 훈련, 장비와 지원 등이 충분히 이루어져야 가능하고, 적용되는 기상 업무에 따라 관측 주기도 다르게 구성되어야 한다.

%------------------------------------------------
\subsection{대기 관측 체계}\index{대기 관측 체계}

요구도 부합하는 대기관측을 위해서는 다양한 센서들을 구비한 적절한 관
측기구와 원격탐사 체계가 필요하다. WMO는 전구규모, 지역규모, 국가규
모의 관측을 위해서 필요한 대기관측 체계에 관한 지침을 제공하고 있다.
전구규모의 관측에 부합하기 위해서는 지상 기반의 하부체계와 위성 기반
의 하부체계가 통합된 형태의 대기관측 체계를 구축하여야 한다. 지상 기
반 하부체계는 지상 종관 기상대, 고층관측, 기후관측 등을 포함하며, 위성
기반 하부체계는 기상위성과 자료 송수신 체계를 포함한다. 지역규모와 국
가규모의 관측은 주로 지상 기반의 관측 체계로 구성되어진다. 정보통신
기술의 발달과 함께 다기능 기상관측 센서를 가진 자동관측 체계들을 네트
워크로 연결하는 기상관측 체계가 대세를 이루고 있다.

\subsection{지상 관측 기상 요소들}\index{지상 관측 기상 요소들}

기상관측의 목적과 체계에 따라 다양한 요구 조건들이 주어지지만 WMO에
서 권고하고 있는 일반 지상관측소에서 관측해야 할 대기관측 요소들은 다
음과 같다.

현재 기상
과거 기상
풍향과 풍속
운량
운형
운저 고도
시정
온도
상대습도
기압
강수량
적설
일사/태양복사량
토양온도
증발량

이들 요소들 이외에도 지상관측소 중에서 일부에서는 특정한 센서들을 활
용하여 고층 관측, 토양 습도 관측, 오존 관측, 대기 조성 관측 등이 이루어
지고 있다. 위에서 언급한 관측 요소들 중에서 운형을 제외한 나머지 요소
들을 자동으로 관측할 수 있는 센서들은 이미 개발되어 활용되고 있다.
그러나 센서와 정보통신 기술력을 모두 통합한다고 하더라도 현재 기상, 과
거 기상, 운량과 운고, 적설과 같은 기상 요소들을 완벽하게 망라할 수 있는
대기관측 체계는 존재하지 않기 때문에 결국 사람에 의한 관측은 기술 발전
과 무관하게 매우 중요하고 가장 정확한 방법으로 남아있다.

\subsection{자동 기상 관측소}\index{자동 기상 관측소}

종관, 기후, 항공기상 분야에서 필요한 대부분의 기상요소들은 자동 관측
장비를 통해서 관측할 수 있다. 자동기상 관측체계의 발전과 함께 전체 관
측소에서 순수한 자동기상 관측소가 차지하는 비율이 관측자에 의한 관측
소에 비해서는 여전히 작지만 꾸준하게 증가하고 있는 추세에 있다. 다만,
자동기상 관측소에서 관측된 값들이 대표성과 유용성을 확보하기 위해서는
적절한 관측지점 선정, 주기적인 센서 교환과 정비 활동 등이 반드시 필요

\subsection{관측자의 중요성}\index{관측자의 중요성}
비록 자동기상 관측 체계들이 발달하고 있다고 하더라도 다음과 같은 몇 가
지 이유들로 인해서 관측자들의 역할은 여전히 중요하다.
(a) 종관/기후 관측에 있어서 적절한 관측 도구의 도움을 받아 불확실성을
제거하고 관측 값들이 대표성을 가질 수 있도록 관측이 이루어져야 한
다.
(b) 관측기구들의 상태를 양호하게 유지하고, 관측자료들을 관리하는 등 관
측소가 좋은 상태를 유지할 수 있도록 한다.
(c) 자동 코딩이나 통신망이 갖추어지지 않은 지역에서의 긴급한 관측과 전
송이 필요한 경우가 많다.
(d) 자동화되어 있지 않은 관측소의 주간/월간 단위 기후학적 자료 정리와
기록이 필요하다.
(e) 자동관측 체계는 다양한 이유로 인해서 필요한 관측 요소를 관측하지
못하거나 관측 체계 전체가 고장날 수 있기 때문에 보조 혹은 백업 관
측이 반드시 필요하다.
(f) 다양한 전문적인 관측 요구에 부응할 수 있다. 따라서 관측자는 필요로
한 표준화된 관측을 할 수 있도록 훈련되어지고 검증된 기상 기관에서
자격을 부여받은 사람이어야 한다. 관측자는 관측기구의 사용법에 대
해서 이해하고 있어야 하며 관측기구를 통해 필요한 관측 요소들을 적
절하게 관측할 수 있도록 훈련되어야 한다.

\subsection{관측소의 선정}\index{관측소의 선정}
관측소 선정의 가장 큰 기준은 관측값이 대표성을 가지도록 하는 것이다.
종관 관측망에 포함되어 있는 관측소는 종관 규모 요구도에 부합하도록 선
정되어야 하며, 항공기상을 지원하는 관측소는 국지(공항)에 특화된 특수
한 조건들에 부합하도록 위치가 선정되어야 대표성을 확보할 수 있다. 관측
소의 설치와 구성에 관한 내용은 WMO 지침에 상세하게 규정되어 있는데
지역규모 혹은 국가규모 관측망에 포함되는 전형적인 종관규모 관측소는
다음과 같은 선정 기준이 제시되고 있다.

(a) 외부에 설치되는 관측기구들은 지상으로부터 일정한 높이에 설치되어
야 하며, 전체 관측 야장은 25m×25m 보다 큰 규모를 권장하지만 부
득이한 경우에는 최소한 10m×7m 까지 축소할 수 있다. 관측소의 바
닥은 짧은 잔디나 주변 지표면 특성을 대표할 수 있는 토양으로 구성되
어 있어야 하고, 관측 장비가 설치된 구역은 바람이 통하는 울타리를
설치하여 비인가 인원의 출입을 막아야 한다. 관측 구역 내에 2m×2m
의 맨 땅을 만들어 지표면 상태와 토양 온도를 관측할 수 있도록 하여
야 하는데 토양 온도는 20cm 미만의 깊이에서 관측하여야 한다.

(b) 관측소 주변에는 가파른 지형이 없어야 하며 우묵한 곳이나 공동 구역
에 설치하지 않아야 하는데 이러한 규칙이 지켜지지 않으면 관측소는
대표성을 확보할 수 없다.
(c) 관측소는 나무, 건물, 벽, 장애물 등으로부터 충분히 멀리 떨어져 있어
야 한다. 우량계는 울타리를 포함한 우량계 주변의 장애물이 우량계 높
이보다 2배 이상 떨어져 있도록 설치되어야 하는데 가능하면 4배 이상
떨어져 있는 것이 좋다.
(d) 일사/복사 측정기, 우량계, 풍향풍속계는 동일한 장소에 서로 충분히 노
출되도록 설치되어야 한다.

(e) 풍향풍속계는 장애물에 의해서 왜곡이 심할 가능성이 높기 때문에 어
떠한 경우에도 장애물로 둘러싸여 있어서는 안 되며, 항상 사방이 열린
곳에 설치되어야 한다.
(f) 대부분의 측기는 사방이 완전히 열려있는 것이 유리하지만, 우량계의
경우에는 강한 바람에 의해서 강수량이 왜곡될 수 있기 때문에 바람의
영향을 적게 받도록 어느 정도의 엄폐 장치가 필요하다.
(g) 일사나 복사량 관측에 있어서 측기가 나무나 건물 등과 같은 장애물로
가려져 수평적인 관측이 크게 제한을 받은 경우 충분한 관측이 가능하
도록 대체 장소를 찾아야 한다.
(h) 구름이나 시정을 관측하는 지점은 가능한 사방의 관측이 가능한 열린
곳이어서 주변 지역과 하늘을 충분히 관측할 수 있는 곳이어야 한다.
(i) 해안에 설치되는 관측소는 바다가 보이는 방향으로 설치하는 것이 좋
다. 그러나 관측소가 너무 해안에서 가깝거나 절벽에 위치하는 경우에
는 장애물에 의한 난류 발생으로 바람과 강수량 관측에 영향을 미칠 수
있다.
(j) 구름과 시정을 야간에 관측할 경우에는 외부 불빛으로부터 영향을 받
지 않는 지역에서 실시하여야 한다.
위에서 언급된 관측소 선정 조건들을 하나하나 따져보면 서로 상치되는 부
분들이 존재한다는 것을 알 수 있다. 따라서 관측소 선정을 할 때는 다양한
요소들을 고려해서 최상의 관측값을 얻을 수 있는 조건을 적절히 조합하여
선정하여야 한다.

\subsection{관측소 좌표}\index{관측소 좌표}
관 측 소 의 위 치 는 세 계 측 지 체 계 1984(WGS-84) 와 지 구 측 지 모 델
1996(EGM96)의 기준에 따라 정확하게 좌표로 지정되어야 한다. 관측소
좌표는 위도와 경도값은 1/1000°까지 상세하게 보고되어야 하며, 관측소
의 높이도 해발고도로 미터(m) 단위로 보고되어야 한다.
관측소의 좌표는 실제 관측이 이루어지는 지점을 의미하며 같은 이름의 도
시나 마을, 공항의 위치를 의미하는 것이 아니다. 종관 관측이 이루어지는
관측소의 높이는 일반적으로 해발고도로 우량계가 서 있는 지역의 높이를
의미한다. 우량계가 없는 관측소는 백엽상의 높이를 의미한다. 공항과 같이
관측소 기압이 매우 중요한 지역의 경우에는 관측소의 높이는 실제 기압
관측이 이루어지는 고도에 맞추는 것이 좋다.

\subsection{측기의 특성과 균질성의 변화}\index{측기의 특성과 균질성의 변화}
관측소의 특성은 나무의 성장이나 주변 건물의 신축 등과 같은 요소들에 의
해서 시간에 따라 계속해서 변화할 수 있기 때문에 관측소는 이러한 환경의
변화가 최소화될 수 있는 지역에 설치되는 것이 좋다. 가능하면 관측소 주
변 환경의 변화를 지속적으로 기록하여 필요한 경우에는 적절한 대체 장소
를 선정하는데 활용하여야 한다.
측기의 교체나 위치를 변경할 경우에는 그 영향이 최소화될 수 있도록 방법
을 강구하여야 한다. 비록 새로운 장비의 물리적인 특성은 잘 알려져 있다
고 하더라도 현업에 활용하기 위해서는 반드시 그 지역의 기후학적 특성에
맞추어 적절하게 보정되어야 한다.
WMO는 새로운 측기를 설치하여 운영하는 경우에 이전 장비를 철수하기
전에 최소한 1년 정도의 비교 관측을 통해서 안정성을 확보하도록 권고하
고 있다. 이러한 기준은 관측 장소를 옮겼을 경우에도 동일하게 적용된다.

측기의 바람직한 특성들

대기관측에 이용되는 측기의 가장 중요한 요구조건들은 다음과 같다.
(a) 불확실성. 특정한 변수들에 대해서 어느 정도의 불확실성을 가지는지
명확해야 한다.
(b) 신뢰성과 안정성
(c) 작동의 편의성, 원활한 보정과 정비
(d) 디자인의 단순함
(e) 내구성
(f) 측기, 소모품, 부속품의 합리적인 유지비용
1.1.4 측기의 일반적인 요구 조건
1.1.4.1 측기의 바람직한 특성들
- 대기관측 및 해석
-
측기가 가지는 계통적인 불확실성, 신뢰성과 안정성은 측기가 일단 설치되
면 오랜 기간 동안 운영 유지되어야 한다는 점에서 가장 중요한 고려 요소
가 된다. 측기를 운영하는 초기에 불확실성 정도에 대한 정확한 정보를 가
지고 있는 것이 측기를 운영하는 전 과정 동안 발생할 수 있는 문제점을 사
전에 인지하는데 유리하다.
일반적으로 측기의 초기보정은 이상적인 값과 관측값 사이의 편차를 줄이
는 과정에서 중요한 의미를 가지며, 측기들은 운영하는 동안 지속적인 정비
활동과 검증 및 보정 작업이 필요하며, 보정 작업이 필요한 시기와 이상 징
후에 대한 정확한 정보가 관측자에게 주어져야 한다.
디자인의 단순함과 내구성은 장비설치, 운영 및 유지에 있어서 중요한 고려
요소가 된다. 왜냐하면 보통 기상 측기는 연중 무중단 운영되며 전문적인
수리가 가능한 지역에서 멀리 떨어진 곳에 설치 운영되는 경우가 많기 때문
이다.

\section{관측 값}\index{관측 값}
\subsection{단위}\index{단위}

관측을 실시하는 목적은 관측 지점의 대기의 상태에 대한 물리적 대기과학
적으로 정량화된 값을 제공하는데 있다. 대기과학적 측면에서는 측기에서
표출되는 변수들을 ‘대기압’, ‘온도’, ‘풍속’ 등으로 표현하지만 대부분의 경
우 앞 쪽에 관측값을 표시하고 바로 이어서 단위를 표시한다(예, 풍속
10m/s). 기상정보는 전 세계에 통용되어야 하기 때문에 대기관측에서 사용
되는 단위들은 국제표준단위(SI)를 사용할 것을 권장하지만 대기과학에서
만 적용되는 단위들을 사용하기도 한다. 그리고 기상 변수들에는 그것이 무
엇을 의미하는지를 기호로 나타내기도 한다. 예를 들어 대기압은 ‘p’를 사
용하여 기압임을 표시한다.
다음의 단위들은 대기관측에서 사용되어야 하는 단위들이다.
(a) 대기압, , 단위 : 헥토파스칼(hPa)
(b) 온도, , 단위 : 섭씨(°C) 혹은 캘빈 온도, , 단위 : 캘빈(K)
(c) 지상풍과 상층풍 풍속, 단위 : 초속(m/s)
(d) 풍향, 불어오는 방향을 표시. 36은 북풍, 09는 동풍을 의미, 단위 : 도(°)
(e) 상대습도, U, 단위 : 퍼센트(%)
(f) 강수량, 단위 : 밀리미터(mm) 혹은 단위 면적당 무게(kg/m^2 )
(g) 강수 강도, Ri, 단위 : 단위 시간당 밀리미터 혹은 단위 시간당 면적당
무게 (kg/m^2/h )
(h) 강설량, 단위 : 단위 면적당 무게(kg/m^2 )
(i) 증발량, 단위 : 밀리미터(mm)
(j) 시정, 단위 : 미터(m)
(k) 조도, 단위 : 단위 면적 당 와트(W/m^2 )
(l) 복사량, 단위 : 단위 면적 당 주울(J/^2 )
(m) 일사 시간, 단위 : 시간(h)

\subsection{상수}\index{상수}
다음은 대기관측에서 사용되는 상수들이다.
(a) 어는 점(T_0)에서의 절대 온도 = 273.15K ( = 0.00°C)
(b) 물의 삼상변화 절대 온도 = 273.16 K ( = 0.01°C)
(c) 표준 중력가속도(g_n ) = 9.80665(m/s^2 )
(d) 0°C에서의 수은 농도 = 1.359 51 × 104 (kg/m^3 )
(n) 운고, 단위 : 미터(m)
(o) 운량, 단위 : 옥타(Oktas)
(p) 상층 지위 고도, 단위 : 지위고도 (gpm)

\subsection{관측 자료 조건}\index{관측자료 조건}
\subsubsection{기압}\index{기압}

주어진 표면에서의 대기의 압력(이하 기압)은 표면 상공의 공기의 무게에
의해서 단위 면적당 가해지는 힘으로 정의된다. 따라서 기압은 지표면에서
대기의 상단까지 뻗어있는 공기 기둥의 무게와 같다. 실제 기압과 별도로
기압의 변화 경향도 잘 측정되어야 한다. 기압변화 경향은 관측 시간 바로
이전 3시간 동안의 기압 변화량을 의미한다. 기압변화 경향은 기압 자체의
변화와 기압변화 특성으로 나누어질 수 있다. 기압 자체의 변화는 문자 그
대로 일정한 시간 간격을 두고 관측된 처음과 끝의 기압 값의 차이를 의미
한다. 기압변화 특성은 일정한 시간 동안 기압이 어떻게 변화해 왔는지에
대한 표시를 의미한다. 예를 들어 기압이 지난 3시간 동안 하강한 후에 상
승했는지, 서서히 상승하다가 급하게 상승했는지가 기압변화 특성이 된다.


기압 측정의 기본 단위는 파스칼( 혹은 N/m^2 )이다. 일반적으로 대기관측
에서 기압값은 앞에 100을 의미하는 두문 ‘헥토(hecto)’를 붙여서 보고하
는데 이때 사용되는 단위는 헥토파스칼(hPa)이 된다. 따라서 기압 1 hPa 은 기압 100 hPa \을 의미하는 것이다. 기존에는 기압의 단위로 많이 사용되었던 밀리바(mb)와 hPa은 같은 값으로 볼 수 있다. 아직까지 몇몇 기 압계들은 그 눈금 단위를 밀리바나 수은주 높이(mmHg)로 표시한 것들을 발
견할 수 있다. 표준 대기상태에서 수은주는 기압이 1013.250 일 때
760(mmHg)의 높이를 가지게 된다.
기압의 단위는 다음과 같이 변환할 수 있다.
1 hPa = 0.750062(mmHg), 
1(mmHg ) = 1.333224 hPa
일반적으로 공학에서 사용되는 밀리미터와 인치 사이에는 1 in = 25.4mm
의 관계식을 가지게 되므로 다음과 같이 변환할 수 있다.
1 hPa = 0.029530(inHg)
1(inHg ) = 33.8639 hPa
1(mmHg) = 0.03937008(inHg)

분석된 기압장은 대기과학적인 측면에서 볼 때 가장 필요한 요소이다. 기압
장은 현재의 대기의 상태를 분석을 통해서 예측으로 이어지는 일련의 기상
업무에서 가장 기본적인 요소이므로 반드시 관측이 이루어져야 한다. 기압
관측은 기술이 허용하는 한 정확하여야 하며, 관측 도구들은 균질한 관측이
보장될 수 있도록 철저하게 보정이 이루어진 후에 관측에 투입되어야 한다.
대기관측에서 사용되는 수은 기압계의 눈금은 바로 실제 기압을 표준 단위
로 읽을 수 있도록 눈금으로 표시되어야 한다. 모든 측기들은 표준 기온
0℃, 중력 가속도 9.80665 에서 정해진 값이 표현되도록 유지되어야
한다.
기압계에는 보통 1개 이상의 눈금이 표시되어 있는데, 과 혹은
과 in 가 동시에 표시되어 있고 표준 대기 상태에서 기압계는 정확하
게 보정되어 사용되어야 한다. 그런데 대기관측에서 항공기상과 같은 특별
한 목적이 있는 경우를 제외하고는 기압값을 보고할 때는 단위로 표시
하여야 한다.
기압관측은 다음과 같은 요구 조건들을 만족하여야 한다.
관측 범위 : 관측소 기압과 해면기압 모두 500 ~ 1080
불확실성(오차) : 0.1 이하
보고 단위 : 0.1
관측 센서 지속 시간 : 20초
자료 표출 주기 : 1분 이하
위에서 언급된 요구조건들을 만족시키기 위해서는 새로운 기압관측 측기
들을 설치할 경우 장비를 현장에 설치하기 이전에 적절한 장비가 구비되고
엄격한 환경에서 검증을 통과한 측기로 보증을 받은 제품을 사용하는 것이
우선되어야 한다. 기압계들은 단독으로 설치 운영되는 경우보다 복합 기상
관측 장비의 일부로 구성되어 있거나, 네트워크로 연결되어 사용될 가능성
이 높기 때문에 장비를 현장에 설치할 때는 사전에 이러한 조건들을 만족
하는지를 강제적으로 확인할 필요가 있다. 또한 장비를 설치 운영하는 동안
지속적인 유지 보수와 보정 작업을 통해서 위의 요구 조건들에 계속해서
부합되도록 관리되어야 한다.

\subsubsection{기온}\index{기온}
WMO는 온도를 물체(고체, 액체, 기체) 속의 분자들의 평균 무작위 운동으
로 특징 지워지는 물리량으로 정의하고 있다. 온도는 두 개의 물체가 어떤
형태로 접촉하여 동등한 온도로 가고자 하는 행위로 특징 지워진다. 따라서
온도는 물체의 열역학적 상태와 두 물체 사이에서의 순수한 열 이동 방향
을 결정하는 물리량으로 표현된다. 두 개의 물체 사이에서 열을 잃는 물체
를 높은 온도라고 표현한다. 그러나 물체의 상태와 관련해서 온도의 물리량
을 정의하는 것은 매우 어려운 일이다. 국제적으로 인정받고 있는 온도 척
도는 물의 삼중점과 어는점을 이용하는 방법이다. 최근 국제적인 공인 온도
척도는 국제온도척도 1990(ITS-90)이다. 대기과학적 온도 범위(-0℃ ~
+60℃)에 대해서 이 온도 척도는 백금의 전기 저항과 273.16K로 정의되
어지는 물의 삼중점과의 선형 관계식을 기초로 하여 만들어진 것이다.
대기관측에 있어서 온도는 다양한 매체를 상대로 측정한다. 가장 일반적인
측정 변수는 다양한 고도에 대한 기온이고, 다른 것들로는 지표면, 토양, 잔

디, 해수면 온도 등이다. WMO는 기온을 “태양의 직접 복사로부터 차폐된
곳의 공기 중에 노출된 온도계가 가리키는 온도”로 정의하고 있다. 비록 이
정의가 그 자체의 열역학적 물리량으로 정의되는 것은 아니지만 광범위 하
게 적용되고 있다.

켈빈 단위(K )를 가지는 열역학적 온도(T ) 혹은 켈빈 온도가 기본 온도가
된다. 켈빈은 물의 삼중점으로부터 열역학적 온도가 1/273.15 씩 변화하는
단위를 말한다. 열역학적 온도에 대응하여 대기과학에서 주로 사용하는 섭
씨() 단위로 나타내는 온도(t )는 아래의 방정식 (1.1)과 같이 정의된다.

t/℃ = T/K - 273.15

섭씨(℃) 1도의 온도 차이는 켈빈 온도(K) 1도 단위와 같다. K 단위는 도(°)
와 함께 표시하지 않는다는 점에 주의해야 한다. 열역학적 온도 척도에 있
어서 온도의 측정은 절대온도 0도(0K)와의 차이로 표현되는데, 절대 온도
0K에서 모든 물질의 분자들은 운동 에너지를 갖지 않는다.
ITS-90에서 사용되는 온도 척도는 몇 가지 물질들의 재생 가능한 평형 상
태에서의 값을 기준으로 하며, 온도 관측 표준 장비들은 이 온도에 맞춰 보
정된다. 국제표준온도 척도는 각 물질들의 이상적인 열역학적 온도와 실제
온도와의 차이를 측정하여 관측기구의 불확실성 정도를 결정한다.


대기관측에서 온도 관측은 다음과 같은 대상을 상대로 실시한다.
(a) 지표면 근처 대기
(b) 지표면
1) 일반 조건

(c) 다양한 깊이의 토양
(d) 해수면과 대형 호수면
(e) 상층 대기
온도 관측들은 공동으로 혹은 단독으로, 국지적으로 혹은 전지구적으로 실
시하며 관측된 온도들은 수치예보 모델 초기자료, 수문학, 농업기상, 기후
변화의 척도 등에 활용된다. 국지적 온도는 전 세계인들의 일일 생리학적
활동에 직접적으로 큰 영향을 미친다. 온도의 측정은 지속적으로 이루어져
야 하며 일정한 시간 간격을 두고 실시되어야 한다.
온도 측정의 범위, 보고 해상도와 불확실성 등에 대한 정보는 1절에서 언급
한 규모에 따라 실시하는 것이 바람직하지만, 실제 관측에 있어서는 모든
조건을 완벽하게 맞출 수 있는 온도계를 구비하는 것이 경제적으로 비효율
적인 측면이 있다. 차라리 싼 온도계들을 표준에 맞추어 보정해서 사용하는
것이 효과적이라고 볼 수 있다.
물론 대기관측에 사용되는 온도계들은 보정 범위와 계산 오차의 크기가 정
해진 기준 이내에 들어있는 것을 사용해야 한다. 또한 온도계의 운영 범위
는 국지적인 기후 범위에 맞추어 선택되어야 한다. 표 1.1은 온도계의 허용
가능한 보정 및 오차 범위를 보여주고 있다.


온도계의 허용 가능한 보정 및 오차 범위

온도계 종료 정규 온도계 최고기온 온도계 최저기온 온도계

온도척도 범위 (˚C) –30 ~ 45 –30 ~ 50 –40 ~ 40
보정 범위 (˚C) –30 ~ 40 –25 ~ 40 –30 ~ 30
최대 허용 오차 <0.2K 0.2K 0.3K

정해진 범위 내에서의 최고, 최저 보정 값들 사이의 차이
0.2K 0.3K 0.5K
10℃ 단위 마다 최대 보정 변위
0.1K 0.1K 0.1K

모든 온도계들은 적절한 불확실성, 운영 기준, 보정 증명서 등을 기록한 같
은 보증서와 함께 제공되어야 한다. 초기 시험과 보정은 국가 기관에서 보
증한 연구실에서 실시되어야 하며, 운용 중인 온도계도 일정한 시간 간격을
두고 지속적으로 체크하고 보정되어야 한다.
일반적인 대기 관측에서 온도계는 아주 작은 시간 상수와 지연계수를 가지
고 있다. 따라서 온도계를 읽을 때 시간 상수를 너무 길게 잡으면 짧은 온도
변화 경향을 평활화 시켜버릴 수 있고, 너무 짧게 잡으면 큰 범위의 온도 변
화 경향을 보는데 불리하다.
보통의 경우 온도계의 감응시간은 20초를 기준으로 삼는다. 온도의 변화는
일일 기상 예보 생산과 지원뿐만 아니라 특히 기후변화 연구에 매우 중요한
자료가 될 수 있다.
온도는 지표면 상태, 식생, 건물의 존재, 백엽상의 모양 등과 같은 주변 환
경의 변화에 매우 민감하게 변화하기 때문에 온도 자료를 저장할 때는 관측
환경의 변화에 관해서도 동시에 기록하는 것이 좋다. 이렇게 어떤 기상 변
수에 대한 정보를 기록하는 것을 ‘자료에 대한 자료(Metadata)’라고 부른
다.

\subsubsection{습도}\index{습도}

WMO는 대기 중의 습도를 나타내는 정의들을 제시하고 있는데, 대기과학
에서 자주 사용하는 대기 중의 습기의 척도를 나타내는 변수들은 다양하다.
혼합비(r) : 건조 공기 질량 대 수증기 질량의 비
비습(q) : 습윤 공기 질량 대 수증기 질량의 비
노점 온도(T_d) : 주어진 기압에서 포화 혼합비와 주어진 혼합비가 같아 습
윤 공기가 포화되는 온도
상대습도 (U) : 같은 온도와 기압에서 물의 포화 수증기압에 대한 관측된
수증기압의 비를 \%로 나타낸 것
수증기압(e' ) : 공기 중의 수증기의 분압
포화 수증기압(e'_w 와 e'_i ) : 물과 얼음의 표면과 공기가 평형을 이루는 상태
의 수증기압

대기 중의 수증기와 관련된 변수들의 단위와 척도는 다음과 같이 정리할 수 있다.
(a) 혼합비(r)와 비습(q) : kg/kg
(b) 공기의 수증기압 (e', e'_w , e'_i) : hPa
(c) 온도(T), 습구 온도(T_w), 노점 온도(T_d), 빙점 온도(T_f) : K
(d) 온도(t), 습구 온도(t_w), 노점 온도(t_d), 빙점 온도(t_f) : ℃
(e) 상대 습도(U) : \%

지표면 부근의 습도 관측은 기상 분석 및 예보, 기후 연구, 수문학, 농업기
상학, 항공기상, 환경 연구 등을 위해서 반드시 필요한 요소이다. 대기 속에
서의 물의 상변화는 기상현상의 발생에서 소멸까지 매우 중요한 변수가 될
수 있다. 습도 관측 시 고려해야할 관측 범위, 해상도, 정확도는 아래 표
4.1과 같다. 표 1.2에서 제시된 정확도들은 작동 상태가 양호하고 정비가
잘 이루어진 관측장비를 기준으로 한 것이다. 여전히 많은 관측소에서는
백엽상에 설치된 단순한 습도계를 이용해서 습도를 관측하기 때문에 운영
성능이 많이 떨어지는 것이 현실이다.

지상 습도 관측 요구 조건들

요구 조건 습구 온도 상대 습도 노점 온도
관측 범위 -10℃ ~ +35℃ 5% ~ 100% -60℃ ~ 35℃
정확도(불확실성)
0.1K(높은 상대습도)
0.2K(보통 상대습도)
1% (높은 상대습도)
5% (보통 상대습도)
0.1K(높은 상대습도)
0.5K(보통 상대습도)
허용 관측 오차 0.2K 3% ~ 5% 0.5K
보고 해상도 0.1K 1% 0.1K
센서 시간 상수 20초 40초 20초
평균 시간 60초 60초 60초

\subsubsection{바람}\index{바람}

풍속은 대규모 조직화된 기류에 편승되어 있고, 공간적, 시간적 무작위 변
동 폭이 상대적으로 작은 3차원 벡터 값으로 표현할 수 있다. 바람을 공기
의 수직적 수평적 흐름의 속도라고 표현할 수 있지만 대기 오염이나 항공기
이착륙 등과 같은 특수한 상황을 제외하고 일반적으로는 풍속은 이차원 벡
터 즉 수직방향과 속도 성분만을 가지는 것으로 정의한다.
일반적 바람과 별도로 빠르게 풍향과 풍속이 변화하는 것을 순간풍(돌풍)
혹은 돌풍성이라고 부른다. 대부분의 경우 바람 자료는 풍향과 풍속이 극좌
표계로 표시된 평균된 바람장이 된다. 좀 더 상세한 바람 정보가 필요할 때
는 바람의 변동성과 돌풍성이 추가로 반영되어야 하는데 이 경우에는 최대
순간풍과 풍속과 풍향의 표준 편차 정보가 포함된다.
평균풍은 10분에서 60분 평균된 수평 바람 정보를 말하는데 보통 기상예보
에 활용하는 경우에는 10분 평균 바람 정보를 이용하고 기후통계 연구에서
는 시간 단위 혹은 주야간 단위 바람장을 활용하고 항공기상 업무에서는 더
짧은 주기의 바람 정보를 필요로 한다. 수 분 단위의 평균된 바람은 자연적
인 난류성 요란을 평활화 시키지 못하기 때문에 ‘1분 평균’ 바람은 긴 순간
풍이라고 표현한다.
순간 최대풍은 일정한 시간 간격 동안 관측된 최대 풍속을 의미한다. 1시간
단위 관측에서 순간 최대풍은 1시간 동안 나타났던 최대 풍속을 의미한다.
순간풍 기간은 순간 최대풍이 관측되었던 기간을 의미하는데, 느리게 반응
하는 관측장비의 경우에는 실제 순간풍을 실제 보다 낮게 나타낼 수 있고
너무 빠르게 반응하는 장비는 짧은 시간에 파동 형태의 풍속 변동을 보여
탁월풍을 한 눈에 알아보기 힘들게 할 수도 있다.

바람 관측은 기상 감시와 예보, 기후, 바람 피해 가능성 탐지, 풍력 에너지
활용 가능성 연구, 지표면 플럭스와 증발량 추정, 오염 물질 확산 추정, 농
업기상 등에 광범위하게 활용된다.
풍속의 정확도는 평균 풍속 5m/s 이하에서는 0.5m/s, 평균 풍속 5m/s 이
상에서는 그 풍속의 10% 미만이면 충분하고 풍향은 5° 범위 내에서 정확
도를 가지고 있어야 하는데, 최근 관측장비들은 이러한 요구 조건들을 충분
히 만족할 만큼 정확도가 뛰어나다.
바람의 관측에서 있어서 장비의 성능보다 더 어려운 문제는 적절한 풍속계
설치 위치를 선정하는 것이다. 넓은 지역의 바람을 대표할 수 있는 관측 장
소를 찾는 것은 거의 불가능하기 때문에 예비 관측을 통해 관측소 오차를
추정하는 작업이 필요하다. 바람의 돌풍성에 관한 정보는 항공기의 이착륙,
풍압 통계, 오염 확산 문제 등에 있어서 매우 중요한 요소이기 때문에 정확
한 정보가 보고되어야 한다.

바람을 관측 할 때는 풍속계를 사용하는데 풍속계는 풍향지시날개와 컵이
달린 형태이거나, 지시날개와 프로펠러가 달린 형태가 사용된다. 그런데 관
측장비가 일시적으로 고장이 나거나 없는 경우에는 풍향과 풍속을 주관적
으로 추정해서 보고해야 한다(표 1.3은 일반적으로 활용되는 풍속 추정 방
법을 정리한 것이다).
바람 관측 센서는 회전식 컵이나 프로펠러에 풍향지시날개가 달린 형태가
일반적인 모습인데, 컵과 지시날개, 프로펠러와 지시날개, 독자적인 프로펠
러 형들이 복합적으로 사용된다. 고전적인 방식으로는 피토관 방식이 있는
데 현재는 일반적인 관측에서 거의 활용되지 않고 있다.

앞에서도 설명되었지만 바람에 대한 정보는 순간적인 정보가 아닌 평균된
바람장과 순간풍 정보를 동시에 포함하고 있어야 하기 때문에 바람 관측장
비는 관측 센서뿐만 아니라 데이터처리와 기록 시스템이 포함되어 있어야
한다. 바람 관측자료를 처리할 때에는 평균, 표준편차 계산 등에서 오류가
발생하지 않도록 해야 하고 순간 최대풍에 대한 정보가 평활화되어 없어지
지 않도록 유의해야 한다. 고전적인 방식이긴 하지만 풍향 풍속 기록계를
계속해서 사용하는 것도 이러한 오류를 최소화하고 관측 기간 내에 발생했
던 바람의 변동을 한눈에 알아볼 수 있기 때문이다.


[표 1.3] 바람 추정 방법
보우퍼트 풍력계급 / 표준 고도 10m에서 관측된 값 / 지상에서 풍속 추정 방법
				kts /  m/s /  km/h
0 고요 <1 0~0.2 <1 고요, 연기가 수직으로 상승
1 실바람 1~3 0.3~1.5 1~5  연기로 바람방향 판단 가능, 풍향계는 움직이지 않음
2 남실바람 4~6 1.6~3.3 6~11  얼굴에 바람을 느낌, 잎이 흔들림, 풍력계 움직임 
3 산들바람 7~10 3.4~5.4 12~19 잎과 잔 가지가 지속적으로 움직임, 작은 깃발 날림
4 건들바람 11~16 5.5~7.9 20~28 먼지나 종이 날림, 나뭇가지 움직임
5 흔들바람 17~21 8.0~10.7 29~38 작은 나무 흔들림, 호수면에 풍랑이 보임
6 된바람 22~27 10.8~13.8 39~49 큰 나뭇가지 움직임, 전선에서 바람소리 발생, 우산 사용 어려움
7 센바람 28~33 13.9~17.1 50~61 모든 나무 흔들림, 바람 속을 걷기 어려움
8 큰바람 34~40 17.2~20.7 62~74 나뭇가지 부러짐, 부산물 발생
9 큰센바람 41~47 20.8~24.4 75~88 약한 건축물 피해 발생
10 노대바람 48~55 24.5~28.4 89~102 나무 뽑힘, 건축물 피해 발생
11 왕바람 56~63 28.5~32.6 103~117 대규모 바람 피해 발생
12 싹쓸바람 > 64 > 32.7 > 118


\subsubsection{구름}\index{구름}

운형을 관측하고 지표면으로부터의 운저고도를 추정하거나 측정하는 것은
여러 가지 목적에서 중요한 요소이다. 특히 항공기상과 응용기상 분야에서
그 중요성은 무시할 수 없다. 기본적으로 관측자와 예보관들은 구름의 형태
와 특성에 대해서 충분히 이해하고 있어야 한다.

구름 : 관측 지점에서 인지가 가능한 아주 작은 수적, 얼음 알갱이 혹은 두
가지의 혼합물들의 집합체로 지표면 상공에 떠 있는 것을 의미한다. 구름
속 수적입자는 반지름이 200 미만으로 반지름이 이것보다 커지면 그 수
적은 이슬비나 비로 추정한다. 아주 특별하고 드물게 나타나는 진주모운이
나 야광운, 성층권 하부에서 가끔 발생하는 권운의 경우를 제외하고는 거의
대부분의 구름의 발생은 대류권 내로 제한된다. 구름은 대류활동, 지형에
의한 강제상승, 저기압, 전선과 같은 대규모 연직 운동 등과 같은 공기의 연
직 운동의 결과로 만들어진다. 구름이 형성에는 적절한 기온감률과 대기 중
습기, 하층 난류 등과 같은 작은 요인들도 작용한다. 기온이 0℃ 이하인 상
운형을 관측하고 지표면으로부터의 운저고도를 추정하거나 측정하는 것은
여러 가지 목적에서 중요한 요소이다. 특히 항공기상과 응용기상 분야에서
그 중요성은 무시할 수 없다. 기본적으로 관측자와 예보관들은 구름의 형태
와 특성에 대해서 충분히 이해하고 있어야 한다.
태에서도 구름 입자들은 전부 수적으로 이루어질 수 있는데 보통 층운의 경
우에는 -0℃까지 대류운의 경우에는 -5℃까지 과냉각 수적의 형태로 존재
한다. 이 온도 기준보다 낮아지면 수적들이 얼음 알갱이로 변하고 구름 속
에는 과냉각 수적과 얼음이 혼합되어 존재하게 된다.
운량 : 구름이 하늘을 덮고 있는 정도를 추정한 값을 운량이라고 하는데, 단
일 운형의 운량과 모든 운형의 운량을 동시에 관측한다. 운량은 보통 하늘
을 8등분으로 나누어 어느 정도로 덮고 있는지를 옥타(Okta) 단위로 표시
하며 구름이 전혀 없는 경우는 0, 구름이 완전히 하늘을 가린 경우는 8로
표시한다.
운저 : 지표면에서 구름 최하부까지의 높이를 운저라고 하는데, 보통 지상
고도를 말한다. 항공기상관측소에서는 지상 고도로 공식적인 공항 해발고
도를 사용한다.
운형은 다양한 방법으로 구분한다.

(a) WMO는 1975년 기본 운형 10종을 결정하고 보다 세부적인 분류는 다
음과 같은 기준을 따른다고 정의하였다.
(i) 구름의 종류(구름의 형태와 구조)
(ii) 구름의 상이성(구름의 배열과 투명도)
(iii) 부가적인 형태와 부속 구름들(모루구름, 유방구름, 꼬리구름, 강
수구름, 아치구름, 깔대기구름, 삿갓구름, 연막구름, 토막구름)
(iv) 원래 구름으로부터 새로운 구름이 만들어지면 ‘파생’이란 의미의
Genitus를 사용해서 나타내고, 원래 구름 자체가 변해서 새로운
형태로 바뀌면 ‘전환’이란 의미의 Mutaus를 붙여서 세부 구름으
로 분류한다. 예를 들면 Stratocumulus Cumulogenitus, Stratus
Stratocumulomutatus 등의 구름 분류가 가능하다.

다양한 구름의 종류를 고도에 따라 상층운, 중층운, 하층운으로 분류할
수도 있다. 대류권 내에서는 상층운은 6~12km(20,000~40,000ft), 중
층운은 지표~6km(0~20,000ft), 하층운은 지표~1.5km(0~5,000ft)로
고도에 따라 분류한다. 상층운은 권운, 권적운, 권층운이 있고, 중층운
은 고적운, 고층운(기준 고도보다 높은 높이에서 발생하기도 함), 난층
운(고도가 낮기도 하고 높기도 함)이 있고, 하층운은 층적운, 층운, 적
운, 적란운(상층운 고도까지 높게 발달함)이 있다. 종관관측에서는 분
- 대기관측 및 해석 -
류 기준에 따라 9종을 사용하는데, 상층운, 중층운, 하층운으로 분류하
고 각각 CH, CM, CL의 부호를 사용해서 나타낸다. 종관규모에서는 각
구름의 형태보다는 전체적인 하늘 상태가 주가 되기 때문에 보다 간략
한 분류법을 사용한다.

수직시정 : 관측자가 자신의 위치에서 아래쪽이나 위쪽에 위치한 물체를 식
별할 수 있는 최대 거리를 의미한다. 구름이 끼어 있는 상태에서 수직 시정
은 목측에 의존하기 힘들기 때문에 빛이나 전자기파를 수직으로 방사해서
얻어지는 연직 프로파일을 통해서 구하는 것이 일반적이다.

구름의 높이는 미터를 주로 사용하고 항공기상에서는 피트를 사용한다. 구
름의 운량에 대한 단위는 옥타(okta)를 사용하는데 이 값은 관측자가 하늘
을 보았을 때 둥글게 보이는 하늘을 8등분하여 구름이 덮인 양을 의미한다.



1) 운량
가장 효과적이고 많이 사용되는 운량 관측은 관측자의 목측에 의한 것이다.
관측장비를 사용하는 방법은 여전히 개발 중이지만 알고리즘을 이용하여
하층운에 대한 운량을 추정하는 정도에 그치고 있다. 운량은 식별되는 층의
운량과 전체 층의 총운량이 동시에 보고되어야 한다. 총운량은 보이는 모든
구름들을 모았을 때 전체 하늘을 가리는 정도를 옥타로 표현한 것이다. 층
별 부분 운량은 동일한 구름의 형태나 구름의 층이 하늘을 가리는 정도를
의미하는데 부분 운량을 모두 합치면 총운량이나 8옥타를 넘을 수도 있다.
WMO가 정한 운량에 대한 코드는 다음과 같다.

WMO 코드 8등분(옥타) 10등분(1/10)
0 0 구름 없음 0 구름 없음
1 1 옥타 미만 1/10 미만
2 2 옥타 2/10 ~ 3/10
3 3 옥타 4/10
4 4 옥타 5/10
5 5 옥타 6/10
6 6 옥타 7/10 ~ 8/10
7 7 옥타 이상 8옥타 미만 9/10 이상 10/10 미만
8 8 옥타 10/10
9 하늘이 안개와 같은 다른 기상 현상으로 인해서 완전히 가려진 상태
/ 하늘이 기상현상이 아닌 다른 이유로 식별할 수 없거나, 관측을 하지 못한 경우

2) 운저고도
구름의 운저고도는 항공기상을 비롯한 기상업무에 매우 중요한 요소이기
때문에 대부분 관측장비를 사용해서 관측하지만 여전히 관측자에 의한 목
측에 의존하는 경우도 많다. 구름 관측을 보고할 때는 관측장비의 정확도에
대한 정보가 같이 포함되어야 하는데 특히 항공기상에서 운저 고도는 항공
기 운항 조건을 결정하는 요소이므로 매우 높은 정밀도를 요구한다. 운저고
도 관측장비는 빛, 풍선, 레이저 등을 구름으로 방사해서 구름의 높이를 측
정한다.


3) 운형
운량과 마찬가지로 현재까지 운형을 파악할 수 있는 유일한 수단은 관측자
의 목측 밖에 없다. 관측자들은 운형 사진과 운형 도감을 활용해서 다양한
형태의 운형에 대해서 인지하고 있어야 한다.

\subsubsection{강수}\index{강수}

강수량계(혹은 우량계)는 가장 많이 사용되는 강수 관측장비이다. 일반적으
로 우량계는 상부에 강수를 모으는 용기가 있고, 아래에 모인 강수량을 측
정할 수 있는 원통 형태의 실린더를 가지는데 전체적인 모양은 깔때기 형태
이다.
우량계의 높이나 구멍의 크기 등은 국가나 기관에 따라 다르고 명확하게 정
해진 표준 규격은 없다. 다만, 우량계의 높이는 수집구가 최대 적설 높이보
다 높아야 하며, 강수 입자가 지표면을 맞고 튕겨 오르는 높이보다는 높아
야 하는데 보통 50cm에서 1m 정도의 높이를 가진다. 강수 측정은 설치 장
소와 바람에 따라 큰 차이를 보일 수 있다(그림 1.4 참조). 전자광학 기술의
발전과 함께 광학적 산란을 이용한 광학 우량계가 현업 기관에서 운영되기
시작했지만 대부분은 기존 우량계와 병행해서 사용한다.
지점별 강수 관측은 지역 강수 분석의 기초적인 자료를 제공하지만 지점 관

측은 제한된 지역만을 대표하고 누적 기간의 길이와 자연지리학적 균일성,
지형, 강수 과정 등에 크게 좌우되는 한계를 가진다. 최근에는 레이더와 위
성 자료를 이용하여 정량적 강수량과 공간적인 분포를 결정하는 기술이 크
게 발전하고 있다. 강수량 자료의 객관성과 활용성을 높이기 위해서는 지점
별 관측 보다는 네트워크를 활용한 강수 관측을 통해 지역 강수 추정의 정
확성을 향상시켜야 한다.
강수 관측은 우량계의 설치 위치에 따라 민감한 차이를 보이기 때문에 강수
관측자료에는 우량계 주변의 특이한 장애물의 수직각, 우량계의 구성, 우량
계 입구의 높이, 바람 관측장비의 높이 등과 같은 관측 환경에 대한 부가적
인 정보가 포함되어야 한다.한 지점에서의 다른 형태의 우량계 사용, 우량
계 위치나 높이의 변화 등은 강수 관측자료의 시간적 불균일성을 가져올 수
있고, 지점 별로 다른 형태의 우량계 사용, 설치 특성의 차이는 공간적인 불
균일성을 가져와 강수 관측의 체계 오차를 만들어 낼 수 있는데 가장 큰 요
인은 바람에 의해서 강수 입자가 수평적으로 날리는 현상이다.

